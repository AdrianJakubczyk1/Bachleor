%%%%%%%%%%%%%%%%%%%%%%%%%%%%%%%%%%%%%%%%%
% Szablon pracy dyplomowej
% Wydział Informatyki 
% Zachodniopomorski Uniwersytet Technologiczny w Szczecinie
% autor Joanna Kołodziejczyk (jkolodziejczyk@zut.edu.pl)
% Bardzo wczesnym pierwowzorem szablonu był
% The Legrand Orange Book
% Version 5.0 (29/05/2025)2023)
%
% Modifications to LOB assigned by %JK
%%%%%%%%%%%%%%%%%%%%%%%%%%%%%%%%%%%%%%%%%
\thispagestyle{empty}
\listoffigures

\chapter{Dodatek}
\label{chapter:dodatek_A}

\section*{Cel}
Niniejszy aneks zawiera zwięzłe, implementacyjne podsumowanie aplikacji webowej zrealizowanej w ramach przedstawionej pracy. System umożliwia udostępnianie, wyszukiwanie i ocenę materiałów edukacyjnych oraz pokazuje, w jaki sposób baza danych w pamięci Hazelcast może poprawić responsywność i skalowalność aplikacji opartej na Spring Boot.

\section*{Przegląd systemu}
Aplikacja stosuje warstwowy, zorientowany na usługi styl projektowy. Warstwa back-end jest zaimplementowana w Javie (Spring Boot) i udostępnia REST-owe endpointy dla interfejsu użytkownika. Hazelcast wykorzystywany jest jako baza danych.

\begin{table}[h]
  \centering
  \caption{Główne komponenty i odpowiedzialności}
  \label{tab:components}
  \begin{tabular}{@{}p{3.3cm}p{9.2cm}@{}}
  \toprule
  Komponent & Odpowiedzialność \\
  \midrule
  Kontrolery & Endpointy HTTPS dla autentykacji, materiałów, wyszukiwania oraz funkcji użytkownika/profilu. \\
  Auth \& Security & Weryfikacja , obsługa sesji/JWT, kontrola dostępu oparta na rolach (user/admin). \\
  Wyszukiwarka & Wyszukiwanie słów kluczowych po zcache'owanych metadanych i tagach. \\
  Rating/Comment Service & feedback od użytkowników. \\
  Admin & przegląd nowych zgłoszeń, akceptowanie i odrzucanie zgłoszeń, usuwanie/edytowanie treści. \\
  \bottomrule
  \end{tabular}
\end{table}

\section*{Wybrane przypadki użycia}
\begin{itemize}
  \item \textbf{Rejestracja użytkownika:} Gość podaje imię, e-mail i hasło; system waliduje dane, sprawdza unikalność e-maila.
  \item \textbf{Logowanie i sesja:} Istniejący użytkownik uwierzytelnia się; wydawana jest sesja.
  \item \textbf{Przesyłanie materiału:} Zalogowany użytkownik wgrywa plik z tytułem i opisem.
  \item \textbf{Przeglądanie artykułów:} Każdy użytkownik może przeglądać stronnicowane listy artykułów.
  \item \textbf{Wyszukiwanie materiałów:} Użytkownik wyszukuje po słowach kluczowych; usługa odpytuje indeks w pamięci, zwracając pasujące zasoby.
  \item \textbf{Podgląd/Pobranie:} Użytkownik otwiera stronę z dostępnym materiałem do pobrania klika przycisk pobierz, udostępniony plik zostaje pobrany
  \item \textbf{Ocena i komentarz:} Pod artykułami użytkownicy wystawiają oceny (1--5) oraz mogą publikować komentarze.
  \item \textbf{Zarządzanie profilem:} Użytkownik aktualizuje dane na stronie do zarządzania profilem
  \item \textbf{Moderacja administracyjna:} Administrator zatwierdza/odrzuca oczekujące zgłoszenia do klas oraz zarządza użytkownikami i artykulami.
\end{itemize}

\section*{przykład REST API}
\noindent Poniższa tabela szkicuje przykładowe REST endpointy. O ile nie wskazano inaczej, wymagane jest uwierzytelnienie.

\begin{table}[h]
  \centering
  \caption{Przykładowe punkty końcowe}
  \label{tab:endpoints}
  \begin{tabular}{@{}p{2.1cm}p{6.4cm}p{3.9cm}@{}}
  \toprule
  Metoda & Ścieżka & Cel / Autoryzacja \\
  \midrule
  POST & \texttt{/api/auth/register} & Utworzenie konta, dostępne dla każdego. \\
  POST & \texttt{/api/auth/login} & Uzyskanie sesji/JWT, dostępne dla każdego. \\
  GET & \texttt{/api/posts} & przegląd postów, dostępne dla każdego. \\
  GET & \texttt{/api/posts/\{id\}} & szczegóły posta; dostępne dla każdego. \\
  POST & \texttt{/api/posts} & Przesłanie nowego materiału, dostępne dla nauczyciela i admina. \\
  GET & \texttt{/api/search?q=...} & Wyszukiwanie postów; dostępne dla każdego. \\
  POST & \texttt{/api/posts/\{id\}/ratings} & Dodanie oceny, dostępne dla zalogowanych. \\
  POST & \texttt{/api/posts/\{id\}/comments} & Dodanie komentarza; dostępne dla zalogowanych. \\
  \bottomrule
  \end{tabular}
\end{table}

\section*{Konfiguracja — wycinek (Hazelcast)}
\noindent Poniższe zestawienie pokazuje minimalny kod konfiguracji Spring Boot uruchamiający instancję Hazelcast:

\begin{lstlisting}[language=yaml,caption={Minimalny fragment konfiguracji Hazelcast},label={lst:hz-config}]
# application.yml
spring:
  cache:
    type: hazelcast
  session:
    store-type: none
# przyklad konfiguracji
# @Bean
# public HazelcastInstance hazelcastInstance() {
#   Config cfg = new Config();
#   cfg.setClusterName("edu-share-cluster");
#   cfg.addMapConfig(new MapConfig("materials")
#       .setTimeToLiveSeconds(0)     
#       .setBackupCount(1));          
#   cfg.addMapConfig(new MapConfig("sessions")
#       .setTimeToLiveSeconds(1800)); # 30 min TTL sesji, jeśli używane
#   return Hazelcast.newHazelcastInstance(cfg);
# }
\end{lstlisting}

\section*{Atrybuty jakości i uzasadnienie}
\textbf{Wydajność i skalowalność:} Ścieżki z przewagą odczytów, korzystają z dostępu w pamięci, co skraca czasy odpowiedzi i odciąża bazę. Skalowanie horyzontalne wspiera dystrybucja danych i klaster Hazelcast

\noindent\textbf{Bezpieczeństwo:} Stosowane są standardowe wzorce Spring Security (haszowanie haseł, autoryzacja oparta na rolach). Operacje wrażliwe (moderacja) wymagają uwierzytelnienia i odpowiednich ról.

\noindent\textbf{Utrzymywalność:} Wyraźny podział odpowiedzialności (kontrolery, serwisy, repozytoria) oraz dobrze zdefiniowane REST-owe endpointy ułatwiają testowanie i przyszłą rozbudowę (np. nowe typy treści).

\section*{Kierunki rozwoju}
Planowane usprawnienia obejmują: lepsza wyszukiwarka, podgląd treści, zaawansowane strategie cache'owania optymalizujące zużycie pamięci przy dużym obciążeniu.

