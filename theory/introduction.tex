%%%%%%%%%%%%%%%%%%%%%%%%%%%%%%%%%%%%%%%%%
% Plik z wstępem do pracy
% Szablon pracy dyplomowej
% Wydział Informatyki 
% Zachodniopomorski Uniwersytet Technologiczny w Szczecinie
% autor Joanna Kołodziejczyk (jkolodziejczyk@zut.edu.pl)
% Bardzo wczesnym pierwowzorem szablonu był
% The Legrand Orange Book
% Version 5.0 (29/05/2025)
%
% Modifications to LOB assigned by %JK
%%%%%%%%%%%%%%%%%%%%%%%%%%%%%%%%%%%%%%%%%


\chapter*{Wstęp}

Współczesne technologie informacyjne oferują szerokie możliwości w zakresie udostępniania materiałów edukacyjnych, co znacząco wpływa na efektywność procesu kształcenia. Wraz z rozwojem technologii webowych oraz frameworków aplikacyjnych, takich jak Spring, powstają zaawansowane aplikacje umożliwiające wygodne zarządzanie i wymianę treści edukacyjnych.

Niniejsza praca przedstawia projekt oraz implementację aplikacji internetowej stworzonej przy użyciu frameworka Spring, której podstawowym celem jest umożliwienie użytkownikom wygodnego dzielenia się materiałami edukacyjnymi, zarządzania nimi, a także analizowania aktywności użytkowników i oceniania materiałów wysyłanych przez użytkowników. W ramach projektu wykorzystane zostały dwie różne bazy danych: PostgreSQL oraz wbudowana baza danych H2. Baza danych PostgreSQL pełni rolę trwałego kontenera danych przechowującego główne informacje aplikacji, takie jak dane użytkowników, treści postów oraz komentarze. Natomiast baza danych H2 została zastosowana do celów przechowywania danych tymczasowych, takich jak statystyki użytkowania aplikacji, dane sesji oraz do testów integracyjnych i jednostkowych.

Praca zawiera opis wykorzystanych technologii, szczegóły implementacyjne poszczególnych modułów aplikacji oraz prezentację rezultatów działania. Omówione zostały również praktyczne aspekty zarządzania danymi, kwestie bezpieczeństwa oraz wydajności aplikacji webowych opartych na Springu. Całość dopełniają wyniki testów oraz podsumowanie możliwości dalszego rozwoju systemu.


\begin{enumerate}
\item Opis dziedziny jakiej dotyczy praca, ze wskazaniem, że temat pracy jest ważny, bieżący, itp.
\item Jaki problem z dziedziny się rozwiązuje.
\item Cel i teza pracy
\item W jaki sposób cel zostanie osiągnięty a tez potwierdzona.
\item Struktura pracy.
\end{enumerate} 