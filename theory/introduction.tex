%%%%%%%%%%%%%%%%%%%%%%%%%%%%%%%%%%%%%%%%%
% Plik z wstępem do pracy
% Szablon pracy dyplomowej
% Wydział Informatyki 
% Zachodniopomorski Uniwersytet Technologiczny w Szczecinie
% autor Joanna Kołodziejczyk (jkolodziejczyk@zut.edu.pl)
% Bardzo wczesnym pierwowzorem szablonu był
% The Legrand Orange Book
% Version 5.0 (29/05/2025)
%
% Modifications to LOB assigned by %JK
%%%%%%%%%%%%%%%%%%%%%%%%%%%%%%%%%%%%%%%%%


\chapter*{Wstęp}


Współczesne aplikacje webowe stawiają coraz wyższe wymagania w zakresie wydajności i skalowalności, co prowadzi do rosnącej popularności systemów baz danych w pamięci (in-memory). Rozwiązania te, dzięki eliminacji opóźnień związanych z operacjami dyskowymi, oferują znaczące przyspieszenie dostępu do danych. Termin ,,baza danych w pamięci'' obejmuje  szerokie spektrum technologii, od prostych pamięci podręcznych po złożone siatki danych i rozproszone bazy danych, z których każda posiada unikalne kompromisy architektoniczne. Wybór niewłaściwej technologii może prowadzić do niepotrzebnej złożoności lub braków funkcjonalnych.

W związku z tym, niniejsza praca stawia sobie za cel:  \textit{Analiza porównawcza wybranych systemów zarządzania bazami danych w pamięci. Wykorzystanie frameworka Spring oraz wybranej bazy danych w pamięci w aplikacji webowej odostępniającej materiały dydaktyczne.} a zakres pracy to:
1. Porównanie popularnych systemów zarządzania bazą danych w pamięci.
2. Opracowanie aplikacji webowej udostępniającej materiały dydaktyczne wykorzystującej wybraną bazę danych.
3. Testowanie i analiza wyników.

Praca realizuje dwuetapowy proces. W pierwszej kolejności przeprowadzona zostanie formalna analiza porównawcza czołowych systemów — Memcached, Redis, Apache Ignite oraz Hazelcast — w celu zidentyfikowania i uzasadnienia wyboru optymalnego rozwiązania. Następnie, w oparciu o dokonaną selekcję, zaprojektowana, zaimplementowana i poddana testom zostanie aplikacja webowa.

W kolejnych rozdziałach przedstawiono szczegóły analizy, proces implementacji z wykorzystaniem frameworka Spring Boot oraz wyniki testów wydajnościowych i funkcjonalnych, które empirycznie weryfikują słuszność podjętej decyzji architektonicznej.
