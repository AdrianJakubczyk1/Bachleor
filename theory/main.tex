%%%%%%%%%%%%%%%%%%%%%%%%%%%%%%%%%%%%%%%%%
% Główny plik pracy
% Szablon pracy dyplomowej
% Wydział Informatyki 
% Zachodniopomorski Uniwersytet Technologiczny w Szczecinie
% autor Joanna Kołodziejczyk (jkolodziejczyk@zut.edu.pl)
% Bardzo wczesnym pierwowzorem szablonu był
% The Legrand Orange Book
% Version 5.0 (29/05/2025)
%
% Modifications to LOB assigned by %JK
%%%%%%%%%%%%%%%%%%%%%%%%%%%%%%%%%%%%%%%%%


%----------------------------------------------------------------------------------------
%	PAKIETY ORAZ PLIKI ZAWIERAJĄCE DEFINICJE STYLI I KONFIGURACJĘ OTOCZEŃ LATEX
%----------------------------------------------------------------------------------------
\documentclass[12pt,fleqn,twoside]{book} % JK Rozmiary czcionek są zgodne z wymaganiami uczelni 
% UWAGA! - wydruk jest dwustronny i jest to zabieg zamierzony
% (removed duplicate) biblatex is loaded in structure.tex
\usepackage{graphicx} 
\usepackage{float} 
\usepackage{plantuml} 
\usepackage{listings}          % do wyświetlania kodu
\lstset{                       % ustawienia dla listings
  basicstyle=\ttfamily\small,
  language=Java,
  frame=single,
  breaklines=true
}
\usepackage[utf8]{inputenc}   % kodowanie UTF-8
\usepackage[T1]{fontenc}      % poprawne łamanie słów, polskie znaki
\usepackage[polish]{babel} 
% \usepackage{fix-cm} % Zapewnienie, że wszystkie rozmiary czcionek są dostępne

\input{structure.tex} % JK - Plik zawierający podstawowe elementy konfigurujące układ dokumentu
% UWAGA! - raczej nie będzie potrzeby zmieniania jego struktury

\input{definitions.tex}  % JK - dodatkowe definicje głównie treść strony tytułowej
% UWAGA! - konieczność edycji celem zmiany autora/tematu/dat itp., itd


%----------------------------------------------------------------------------------------
% OTWARCIE DOKUMENTU
%----------------------------------------------------------------------------------------
\begin{document}

%----------------------------------------------------------------------------------------
% STRONA TYTUŁOWA 
%----------------------------------------------------------------------------------------
\include{title_page}

%----------------------------------------------------------------------------------------
% PUSTA STRONA PO STRONIE TYUŁOWEJ (JK - design and implementation)
%----------------------------------------------------------------------------------------
\newpage
\thispagestyle{empty}
~\vfill

%----------------------------------------------------------------------------------------
%	STRESZCZENIE I SŁOWA KLUCZOWE (1 STRONA) (JK - design and implementation)
%----------------------------------------------------------------------------------------
\newpage
\thispagestyle{empty}
%%%%%%%%%%%%%%%%%%%%%%%%%%%%%%%%%%%%%%%%%
% Specjalna strona pracy ze streszczeniem i abstractem w j. angielskim
% Szablon pracy dyplomowej
% Wydział Informatyki 
% Zachodniopomorski Uniwersytet Technologiczny w Szczecinie
% autor Joanna Kołodziejczyk (jkolodziejczyk@zut.edu.pl)
% Bardzo wczesnym pierwowzorem szablonu był
% The Legrand Orange Book
% Version 5.0 (29/05/2025)
%
% Modifications to LOB assigned by %JK
%%%%%%%%%%%%%%%%%%%%%%%%%%%%%%%%%%%%%%%%%


\begin{center}
\noindent {{\color{blueZUT}\Large\sffamily  {Streszczenie}}}\\[1cm] 



Celem pracy było przeprowadzenie analizy porównawczej wiodących systemów baz danych w pamięci oraz, na jej podstawie, opracowanie aplikacji w języku Java udostępniającej materiały dydaktyczne.
W wyniku analizy systemów Memcached, Redis, Apache Ignite i Hazelcast, do implementacji wybrano ten ostatni, w którym wszystkie dane (lekcje, zadania, użytkownicy) przechowywane są w klastrowych mapach pamięciowych, co zapewnia niskie opóźnienia i łatwą skalowalność.
W części teoretycznej przedstawiono kryteria oceny i przeprowadzono szczegółowe porównanie architektur, modeli danych i wydajności analizowanych systemów.
Część praktyczna obejmuje projekt i implementację aplikacji w oparciu o Spring Boot 3.4.1, realizację operacji CRUD na strukturach `IMap`, a także budowę paneli administratora i nauczyciela oraz mechanizmów interakcji użytkowników.
Przeprowadzone testy wydajnościowe i integracyjne potwierdziły słuszność wyboru architektonicznego, wykazując wysoką wydajność i poprawność działania rozwiązania.
Zaproponowana aplikacja stanowi skalowalną podstawę do dalszego rozwoju.


\end{center}

 %abstract.tex 
 
%----------------------------------------------------------------------------------------
%	SPIS TREŚCI (JK - design and implementation)
%----------------------------------------------------------------------------------------
\pagestyle{empty}  % Wyłącz stopkę i nagłówek w TOC
\tableofcontents % wyświetl spis
\addtocontents{toc}{\protect\thispagestyle{empty}} % Zachowaj pusty nagłówek i stopkę w TOC
% Wymuszenie rozpoczęcie pierwszego rozdziału na nieparzystej stronie, aby znajdował się po prawej stronie 
\cleardoublepage 
% Ponownie włącz nagłówki i stopki
\pagestyle{fancy} 

%----------------------------------------------------------------------------------------
%	WSTĘP W PRACY DYPLOMOWEJ (JK - design and implementation)
%----------------------------------------------------------------------------------------
\addcontentsline{toc}{chapter}{Wstęp}
%%%%%%%%%%%%%%%%%%%%%%%%%%%%%%%%%%%%%%%%%
% Plik z wstępem do pracy
% Szablon pracy dyplomowej
% Wydział Informatyki 
% Zachodniopomorski Uniwersytet Technologiczny w Szczecinie
% autor Joanna Kołodziejczyk (jkolodziejczyk@zut.edu.pl)
% Bardzo wczesnym pierwowzorem szablonu był
% The Legrand Orange Book
% Version 5.0 (29/05/2025)
%
% Modifications to LOB assigned by %JK
%%%%%%%%%%%%%%%%%%%%%%%%%%%%%%%%%%%%%%%%%


\chapter*{Wstęp}

Współczesne technologie informacyjne oferują szerokie możliwości w zakresie udostępniania materiałów edukacyjnych, co znacząco wpływa na efektywność procesu kształcenia. Wraz z rozwojem technologii webowych oraz frameworków aplikacyjnych, takich jak Spring, powstają zaawansowane aplikacje umożliwiające wygodne zarządzanie i wymianę treści edukacyjnych.

Niniejsza praca przedstawia projekt oraz implementację aplikacji internetowej stworzonej przy użyciu frameworka Spring, której podstawowym celem jest umożliwienie użytkownikom wygodnego dzielenia się materiałami edukacyjnymi, zarządzania nimi, a także analizowania aktywności użytkowników i oceniania materiałów wysyłanych przez użytkowników. W ramach projektu wykorzystane zostały dwie różne bazy danych: PostgreSQL oraz wbudowana baza danych H2. Baza danych PostgreSQL pełni rolę trwałego kontenera danych przechowującego główne informacje aplikacji, takie jak dane użytkowników, treści postów oraz komentarze. Natomiast baza danych H2 została zastosowana do celów przechowywania danych tymczasowych, takich jak statystyki użytkowania aplikacji, dane sesji oraz do testów integracyjnych i jednostkowych.

Praca zawiera opis wykorzystanych technologii, szczegóły implementacyjne poszczególnych modułów aplikacji oraz prezentację rezultatów działania. Omówione zostały również praktyczne aspekty zarządzania danymi, kwestie bezpieczeństwa oraz wydajności aplikacji webowych opartych na Springu. Całość dopełniają wyniki testów oraz podsumowanie możliwości dalszego rozwoju systemu.


\begin{enumerate}
\item Opis dziedziny jakiej dotyczy praca, ze wskazaniem, że temat pracy jest ważny, bieżący, itp.
\item Jaki problem z dziedziny się rozwiązuje.
\item Cel i teza pracy
\item W jaki sposób cel zostanie osiągnięty a tez potwierdzona.
\item Struktura pracy.
\end{enumerate}  % introduction.tex zawiera treść wstępu

%----------------------------------------------------------------------------------------
%	ROZDZIAŁ 1
%----------------------------------------------------------------------------------------
%%%%%%%%%%%%%%%%%%%%%%%%%%%%%%%%%%%%%%%%%
% Szablon pracy dyplomowej
% Wydział Informatyki 
% Zachodniopomorski Uniwersytet Technologiczny w Szczecinie
% autor Joanna Kołodziejczyk (jkolodziejczyk@zut.edu.pl)
% Bardzo wczesnym pierwowzorem szablonu był
% The Legrand Orange Book
% Version 5.0 (29/05/2025)
%
% Modifications to LOB assigned by %JK
%%%%%%%%%%%%%%%%%%%%%%%%%%%%%%%%%%%%%%%%%

%----------------------------------------------------------------------------------------
%	CHAPTER 1
% 	author: Joanna Kolodziejczyk (jkolodziejczyk@zut.edu.pl)
%----------------------------------------------------------------------------------------

\chapter{Technologie zastosowane w projekcie}
\label{rozdzial1}

Celem niniejszego rozdziału jest przedstawienie technologii oraz narzędzi wykorzystanych w procesie implementacji aplikacji do udostępniania materiałów edukacyjnych. W rozdziale opisano przede wszystkim technologię Spring Framework oraz zastosowaną bazę danych Hazelcast.

\section{Język Java i Platforma JVM}

Java to język ogólnego przeznaczenia o statycznym typowaniu, którego programy uruchamia \emph{Java Virtual Machine} (JVM). Kompilacja do bajtkodu i wykonanie na JVM sprawiają, że ten sam artefakt może działać na różnych systemach operacyjnych bez rekompilacji.  \cite{java-docs}.

\subsection*{JDK, standardowa biblioteka i model wykonania}
Środowisko deweloperskie (JDK) dostarcza narzędzia takie jak kompilator \texttt{javac} oraz rozbudowaną bibliotekę standardową. Znajdują się tam m.in.\ kolekcje, strumienie wejścia/wyjścia, API sieciowe. W praktyce oznacza to, że wiele typowych zadań aplikacyjnych można zrealizować bez dodatkowych zależności. \cite{java-docs,java-head-first}.

\subsection*{Zarządzanie pamięcią i bezpieczeństwo}
JVM automatycznie odzyskuje nieużywaną pamięć poprzez Garbage Collector. Dzięki Garbage Collector nie trzeba się martwić zarządzaniem pamięcią \cite{java-docs}.


\subsection*{Dobre praktyki projektowe}
W codziennej pracy istotne są zasady dotyczące m.in.\ niezmienności, poprawnego nadpisywania \texttt{equals}/\texttt{hashCode} i bezpiecznego publikowania obiektów. Takie wytyczne pomagają unikać subtelnych błędów oraz poprawiają czytelność i utrzymywalność kodu w większych bazach kodu \cite{effective-java-3e}.

\section{Spring Framework}

Spring stanowi szeroko wykorzystywany zestaw komponentów do tworzenia aplikacji w języku Java. Framework dostarcza mechanizmy wstrzykiwania zależności, wsparcie dla warstwowej architektury oraz integracje z powszechnie używanymi bibliotekami. W projekcie zastosowano Spring Boot, który upraszcza konfigurację i uruchamianie usług dzięki zestawowi starterów czyli gotowym modułom zależności, które automatycznie importują niezbędne biblioteki, frameworki i konfiguracje,co istotnie przyspiesza implementację i ułatwia utrzymanie rozwiązania \cite{spring-boot}.

W projekcie wykorzystano między innymi:

– \textbf{Spring MVC} –- do implementacji warstwy widoku i obsługi żądań HTTP.

– \textbf{Spring Security} -– do zarządzania uwierzytelnianiem użytkowników oraz autoryzacją dostępu do zasobów aplikacji.

– \textbf{Spring Data} -– do zarządzania dostępem do baz danych za pomocą wygodnych repozytoriów oraz operacji CRUD.

\section{Baza danych w pamięci Hazelcast}

W projekcie wykorzystano bazę danych w pamięci  – Hazelcast. Baza ta charakteryzuje się wyjątkową wydajnością, niskimi opóźnieniami oraz dużą skalowalnością dzięki możliwości pracy w klastrze. Najczęściej wykorzystywaną strukturą jest rozproszona mapa klucz–wartość (\texttt{IMap}), Hazelcast służy jako szybka warstwa przechowywania i wymiany danych pomiędzy komponentami systemu, przechowująca wszystkie dane aplikacji, takie jak informacje o użytkownikach, lekcjach, zadaniach, komentarzach, ocenach oraz dane sesyjne i statystyczne \cite{hazelcast-docs}. Wybór Hazelcast jako kluczowego komponentu warstwy danych został poprzedzony szczegółową analizą porównawczą alternatywnych rozwiązań, która została przedstawiona w kolejnym rozdziale.

Kluczowe zalety wykorzystania Hazelcast w projekcie:

– Wysoka wydajność -– wszystkie operacje na danych odbywają się w pamięci RAM, co zapewnia szybki dostęp do informacji.

– Skalowalność -– możliwość łatwego dodawania nowych węzłów do klastra w celu zwiększenia wydajności i pojemności systemu.

– Trwałość i odporność na awarie -– dzięki mechanizmowi kopi zapasowych (backup-count), dane są zabezpieczone przed utratą w razie awarii jednego z węzłów klastra.

– Bogate możliwości indeksowania oraz szybkiego wyszukiwania danych.

– Przechowywanie danych sesyjnych i tymczasowych bez konieczności stosowania dodatkowych rozwiązań.

Zastosowanie Hazelcast pozwoliło na znaczące uproszczenie architektury aplikacji oraz zwiększenie jej ogólnej wydajności i niezawodności.

\section{Inne technologie}

Oprócz wymienionych wyżej technologii w projekcie zastosowano dodatkowo kilka narzędzi i bibliotek, które wspomagają budowanie nowoczesnych aplikacji webowych oraz poprawiają jakość i komfort użytkowania.

- \textbf{Thymeleaf} –  silnik szablonów, który integruje się z technologią Spring. Umożliwia renderowanie stron HTML, automatycznie dostarczając dane z modelu aplikacji bezpośrednio do widoków. Ułatwia wiązanie danych z warstwą prezentacji i dobrze współpracuje ze Spring Boot \cite{thymeleaf-docs}.

– \textbf{Bootstrap 5} – popularny framework CSS, który zapewnia podstawowe style oraz komponenty, pozwalając na szybkie tworzenie responsywnych i estetycznych interfejsów użytkownika. Zapewnia między innymi spójne układy i typowe elementy nawigacyjne \cite{bootstrap-docs}.

– \textbf{JUnit} – biblioteka testowa przeznaczona do pisania i wykonywania testów, w projekcie wykorzystywano testy JUnit do sprawdzania poprawności działania poszczególnych komponentów systemu \cite{junit-docs}.


– \textbf{Maven} – narzędzie do zarządzania cyklem życia projektu Java (kompilacja, testowanie, pakowanie) oraz zależnościami, oparte na pliku \texttt{pom.xml} \cite{maven-docs}.

– \textbf{JavaScript} – język programowania stosowany po stronie przeglądarki, w projekcie JavaScript wykorzystywany był przede wszystkim do obsługi interakcji użytkownika, walidacji danych po stronie klienta oraz obsługi asynchronicznych żądań za pomocą technologii AJAX \cite{javascript-docs}.

– \textbf{CSS} – kaskadowe arkusze stylów odpowiedzialne za wygląd i układ elementów interfejsu, stosowane razem z Bootstrap w celu uzyskania spójnej warstwy wizualnej \cite{css-docs}.

W kolejnych rozdziałach przedstawione zostaną szczegóły implementacyjne dotyczące wyżej wymienionych technologii oraz sposób ich wykorzystania w praktyce.






 % chapter1.tex zawiera treść rozdziału 1

%----------------------------------------------------------------------------------------
%	ROZDZIAŁ 2
%----------------------------------------------------------------------------------------
%%%%%%%%%%%%%%%%%%%%%%%%%%%%%%%%%%%%%%%%%
% Szablon pracy dyplomowej
% Wydział Informatyki 
% Zachodniopomorski Uniwersytet Technologiczny w Szczecinie
% autor Joanna Kołodziejczyk (jkolodziejczyk@zut.edu.pl)
% Bardzo wczesnym pierwowzorem szablonu był
% The Legrand Orange Book
% Version 5.0 (29/05/2025)
%
% Modifications to LOB assigned by %JK
%%%%%%%%%%%%%%%%%%%%%%%%%%%%%%%%%%%%%%%%%


%----------------------------------------------------------------------------------------
%	CHAPTER 2
%----------------------------------------------------------------------------------------
\sloppy

\chapter{Opis implementacji systemu}

W tym rozdziale zostaną szczegółowo przedstawione kwestie związane z implementacją aplikacji do udostępniania materiałów edukacyjnych. Omówiona zostanie architektura aplikacji, struktura baz danych, podział aplikacji na moduły oraz najważniejsze aspekty związane z realizacją konkretnych funkcjonalności.

\section{Architektura aplikacji}

Aplikacja została oparta o wzorzec architektoniczny MVC (\textit{Model-View-Controller}), który zapewnia czytelny podział aplikacji na warstwy:

\begin{itemize}
    \item \textbf{Model} – reprezentujący logikę biznesową oraz struktury danych. Realizowany jest za pomocą klas encji oraz repozytoriów Spring Data JDBC.
    \item \textbf{View} – odpowiedzialny za prezentację danych użytkownikom końcowym. Do jego realizacji wykorzystano szablony Thymeleaf oraz biblioteki Bootstrap.
    \item \textbf{Controller} – obsługujący żądania użytkowników oraz zapewniający komunikację pomiędzy widokiem a modelem aplikacji.
\end{itemize}

Takie podejście zapewnia przejrzystą strukturę projektu, ułatwiając jego dalszą rozbudowę oraz utrzymanie.

\section{Struktura baz danych}

W projekcie zastosowano dwa różne systemy bazodanowe:

\subsection{PostgreSQL}

PostgreSQL jest główną bazą danych aplikacji, odpowiedzialną za trwałe przechowywanie danych produkcyjnych. Główne tabele obejmują:

\begin{itemize}
    \item \texttt{users} – przechowuje dane użytkowników aplikacji, takie jak nazwa użytkownika, hasło, email, rola.
    \item \texttt{posts} – zawiera publikowane materiały edukacyjne wraz z ich szczegółami jak tytuł, treść, autor oraz metadane (liczba wyświetleń, liczba komentarzy, liczba polubień).
    \item \texttt{comments} – przechowuje komentarze użytkowników dotyczące materiałów.
    \item \texttt{grades} – przechowuje oceny materiałów przyznawane przez nauczycieli oraz studentów.
\end{itemize}

\subsection{H2 (in-memory)}

Baza danych H2 wykorzystywana jest do celów testowych, przechowywania sesji użytkowników oraz zbierania danych statystycznych dotyczących aplikacji. Główną tabelą wykorzystywaną przez tę bazę jest tabela \texttt{app\_statistics}, zawierająca takie pola jak:

\begin{itemize}
    \item \texttt{id} – unikalny identyfikator rekordu statystyki.
    \item \texttt{stat\_name} – nazwa mierzonego wskaźnika.
    \item \texttt{stat\_value} – wartość statystyki.
    \item \texttt{timestamp} – znacznik czasu utworzenia rekordu.
\end{itemize}

Użycie osobnych baz danych pozwala na klarowny podział odpowiedzialności oraz łatwiejsze zarządzanie danymi w zależności od ich przeznaczenia.

\section{Organizacja kodu źródłowego}

Kod źródłowy aplikacji został zorganizowany zgodnie z zasadami modularności oraz dobrymi praktykami programistycznymi. Struktura projektu dzieli się na moduły:

\begin{itemize}
    \item \textbf{Moduł \texttt{persistent}} – odpowiedzialny za trwałe dane aplikacji (PostgreSQL).
    \item \textbf{Moduł \texttt{temp}} – zawierający klasy odpowiedzialne za zarządzanie danymi tymczasowymi, sesjami i danymi statystycznymi (H2).
    \item \textbf{Warstwa usług (\textit{Services})} – udostępniająca logikę biznesową oraz operacje na danych.
    \item \textbf{Warstwa kontrolerów} – odpowiada za obsługę żądań HTTP oraz zarządzanie widokami.
    \item \textbf{Warstwa widoków (Thymeleaf)} – odpowiedzialna za generowanie dynamicznych stron HTML.
\end{itemize}

\section{Implementacja najważniejszych funkcjonalności}

Do kluczowych funkcjonalności systemu, których implementacja została omówiona w tej pracy należą:

\begin{itemize}
    \item \textbf{System uwierzytelniania i autoryzacji} – oparty o Spring Security, umożliwiający zarządzanie dostępem do zasobów aplikacji zgodnie z rolą użytkowników.
    \item \textbf{Publikacja i zarządzanie materiałami edukacyjnymi} – umożliwia użytkownikom tworzenie, edytowanie i usuwanie materiałów.
    \item \textbf{Komentowanie i ocenianie materiałów} – pozwala na interakcję użytkowników oraz ocenianie publikowanych materiałów edukacyjnych.
    \item \textbf{System statystyk i analiz} – regularnie aktualizowany przez zaplanowane zadania (\textit{Scheduler}), umożliwiający wizualizację danych takich jak najczęściej przeglądane lub komentowane materiały oraz najlepiej oceniani użytkownicy.
\end{itemize}

Każda z tych funkcjonalności została dokładniej omówiona wraz z przykładami kodu źródłowego ilustrującymi kluczowe elementy implementacji.

\section{Testowanie aplikacji}

Testowanie aplikacji zostało oparte o framework JUnit z wykorzystaniem H2 jako bazy danych w pamięci operacyjnej. W projekcie wyróżniono następujące rodzaje testów:

\begin{itemize}
    \item \textbf{Testy jednostkowe} – weryfikują poprawność działania poszczególnych metod oraz logiki biznesowej.
    \item \textbf{Testy integracyjne} – obejmują całe ścieżki żądań HTTP i weryfikują poprawną współpracę poszczególnych modułów.
\end{itemize}

Użycie bazy danych H2 do testów umożliwiło szybkie uruchamianie oraz wysoką izolację testów, co przyspieszyło proces weryfikacji jakości kodu aplikacji.
 % chapter2.tex zawiera treść rozdziału 2

%----------------------------------------------------------------------------------------
%	ROZDZIAŁ 3
%----------------------------------------------------------------------------------------
\chapter{Prezentacja działania aplikacji}

W tym rozdziale zostaną przedstawione praktyczne przykłady działania aplikacji stworzonej do udostępniania materiałów edukacyjnych. Przedstawiono tutaj interfejs użytkownika oraz omówiono realizację kluczowych scenariuszy, takich jak rejestracja, logowanie, zarządzanie materiałami, przeglądanie statystyk, a także administracja aplikacją.

\section{Interfejs użytkownika}

Interfejs użytkownika aplikacji został zaprojektowany z wykorzystaniem biblioteki Bootstrap, która zapewnia responsywność oraz estetyczny wygląd stron. Dynamiczne treści są generowane przez silnik szablonów Thymeleaf, co umożliwia łatwą i szybką integrację warstwy widoku z logiką aplikacji. Przykładowe widoki aplikacji to:

\begin{itemize}
    \item \textbf{Strona główna} – zawiera listę dostępnych materiałów edukacyjnych, wyszukiwarkę oraz przyciski nawigacyjne do logowania i rejestracji.
    \item \textbf{Widok materiału} – prezentuje pełną treść materiału, komentarze użytkowników oraz oceny wystawione przez nauczycieli i studentów.
    \item \textbf{Panel administracyjny} – umożliwia zarządzanie użytkownikami, publikowanymi treściami oraz wyświetlanie szczegółowych statystyk działania aplikacji.
\end{itemize}

\section{Rejestracja i logowanie użytkowników}

Proces rejestracji nowego użytkownika jest intuicyjny oraz wymaga podania podstawowych informacji, takich jak nazwa użytkownika, adres email oraz hasło. Po rejestracji użytkownik może zalogować się do systemu, gdzie uzyskuje dostęp do dodatkowych funkcji, zależnych od jego roli (nauczyciel, uczeń lub administrator). System uwierzytelniania został zaimplementowany z użyciem biblioteki Spring Security, zapewniając wysoki poziom bezpieczeństwa i ochrony danych.

\section{Publikacja oraz zarządzanie materiałami edukacyjnymi}

Po zalogowaniu użytkownik (np. nauczyciel) może publikować materiały edukacyjne, które następnie są dostępne dla innych użytkowników platformy. Proces ten obejmuje:

\begin{itemize}
    \item Tworzenie nowych postów z możliwością dołączania plików.
    \item Edycję istniejących treści oraz zarządzanie ich dostępnością.
    \item Usuwanie materiałów, które nie są już potrzebne.
\end{itemize}

Użytkownicy mogą również komentować materiały oraz oceniać je, co zwiększa interaktywność platformy.

\section{Przeglądanie statystyk}

Aplikacja udostępnia panel statystyk, dostępny dla administratorów, który regularnie aktualizuje dane dotyczące użytkowników i materiałów. Wśród prezentowanych informacji znajdują się m.in.:

\begin{itemize}
    \item Liczba użytkowników i nauczycieli korzystających z aplikacji.
    \item Najczęściej wyświetlane oraz komentowane materiały edukacyjne.
    \item Ranking użytkowników z najwyższymi ocenami materiałów.
\end{itemize}

Statystyki są przedstawione w przejrzysty sposób, z wykorzystaniem wykresów generowanych za pomocą biblioteki Chart.js.

\section{Scenariusze użytkowania aplikacji}

Poniżej zaprezentowano kilka przykładowych scenariuszy użycia aplikacji, wraz z krótkim opisem ich realizacji w praktyce.

\subsection{Scenariusz 1: Dodanie nowego materiału}

Nauczyciel po zalogowaniu się do systemu przechodzi do widoku „Dodaj materiał”, wypełnia formularz oraz opcjonalnie załącza dodatkowy plik. Po wysłaniu formularza materiał jest widoczny na liście wszystkich materiałów.

\subsection{Scenariusz 2: Ocena materiału}

Uczeń przeglądając materiał edukacyjny może wystawić ocenę oraz zostawić komentarz. Informacje te są widoczne dla innych użytkowników, co ułatwia ocenę jakości i przydatności materiału.

\subsection{Scenariusz 3: Analiza popularności materiałów}

Administrator w panelu administracyjnym może szybko sprawdzić, które materiały cieszą się największą popularnością, co pozwala lepiej dostosować publikowane treści do potrzeb użytkowników.

\section{Podsumowanie działania aplikacji}

Zaprezentowane scenariusze działania aplikacji pokazują jej funkcjonalność i użyteczność w praktyce. Platforma jest intuicyjna w obsłudze, zapewnia efektywną wymianę materiałów edukacyjnych oraz umożliwia szybką interakcję między użytkownikami. Dodatkowo funkcje administracyjne pozwalają na stałe monitorowanie aktywności użytkowników oraz dostosowanie treści edukacyjnych do ich potrzeb. % chapter3.tex zawiera treść rozdziału 3

%----------------------------------------------------------------------------------------
%	ROZDZIAŁ 4
%----------------------------------------------------------------------------------------
\chapter{Testowanie i ocena aplikacji}

W niniejszym rozdziale przedstawiono proces testowania aplikacji oraz ocenę poprawności jej działania. Omówione zostały podejścia wykorzystane w testach jednostkowych oraz integracyjnych, a także ocena wydajności aplikacji.

\section{Wprowadzenie do testowania aplikacji}

Testowanie stanowi kluczowy element procesu tworzenia oprogramowania, umożliwiający wykrycie błędów oraz weryfikację poprawności działania zaimplementowanych funkcjonalności. W ramach tej aplikacji wykorzystano dwa główne rodzaje testów:

\begin{itemize}
    \item Testy jednostkowe – testujące pojedyncze klasy oraz metody.
    \item Testy integracyjne – sprawdzające poprawność współdziałania wielu komponentów systemu.
\end{itemize}

\section{Testy jednostkowe}

Testy jednostkowe zostały napisane z użyciem frameworka JUnit 5, który jest szeroko stosowany w aplikacjach Java ze względu na łatwość integracji ze środowiskiem Spring Boot. 

Przykładowe testy jednostkowe objęły klasy:

\begin{itemize}
    \item Serwisy (np. \texttt{StatsService}), które weryfikują poprawność operacji na danych.
    \item Klasy DAO (np. \texttt{StatsDaoImpl}), które sprawdzają prawidłowe wykonanie zapytań SQL do bazy danych H2.
    \item Kontrolery, gdzie przy pomocy MockMVC testowano poprawność zwracanych widoków oraz danych.
\end{itemize}

\subsection{Przykład testu jednostkowego}

Poniżej przedstawiono przykład testu jednostkowego dla klasy \texttt{StatsService}:

\begin{lstlisting}[language=Java]
@Test
void shouldUpdateStatisticsCorrectly() {
    when(postRepository.count()).thenReturn(10L);
    when(userRepository.countByRole("TEACHER")).thenReturn(2L);

    statsService.updateStatistics();

    verify(statsDao, times(1)).saveStat(any(AppStatistic.class));
}
\end{lstlisting}

Wykorzystanie mechanizmów \texttt{Mockito} pozwoliło na izolację testowanych metod od zewnętrznych zależności.

\section{Testy integracyjne}

Testy integracyjne przeprowadzono z wykorzystaniem frameworka Spring Boot oraz wbudowanej bazy danych H2. Głównym celem było sprawdzenie poprawności współpracy różnych warstw aplikacji.

Do realizacji tych testów wykorzystano:

\begin{itemize}
    \item \texttt{TestRestTemplate} – do symulacji pełnych zapytań HTTP i sprawdzania zwracanych odpowiedzi.
    \item Spring Test – zapewniający kompleksową konfigurację środowiska testowego.
\end{itemize}

\subsection{Przykład testu integracyjnego}

Poniżej przedstawiono przykład testu integracyjnego sprawdzającego poprawność działania strony głównej aplikacji:

\begin{lstlisting}[language=Java]
@SpringBootTest(webEnvironment = SpringBootTest.WebEnvironment.RANDOM_PORT)
class ApplicationIntegrationTests {

    @Autowired
    private TestRestTemplate restTemplate;

    @Test
    void shouldLoadHomePageSuccessfully() {
        ResponseEntity<String> response = restTemplate.getForEntity("/", String.class);

        assertEquals(HttpStatus.OK, response.getStatusCode());
        assertTrue(response.getBody().contains("Educational Materials Sharing Platform"));
    }
}
\end{lstlisting}

Testy integracyjne potwierdziły poprawne współdziałanie komponentów aplikacji.

\section{Baza danych testowa (H2)}

W trakcie realizacji testów wykorzystano bazę danych H2 działającą w pamięci, która umożliwiła szybkie i bezpieczne wykonywanie operacji CRUD, bez ingerencji w dane produkcyjne. Baza ta została również użyta do przechowywania sesji użytkowników oraz tymczasowych statystyk aplikacji. 

Przykładowa konfiguracja użytej bazy danych:

\begin{lstlisting}[language=properties]
spring.datasource.url=jdbc:h2:mem:testdb
spring.datasource.driverClassName=org.h2.Driver
spring.datasource.username=sa
spring.datasource.password=
spring.jpa.database-platform=org.hibernate.dialect.H2Dialect
\end{lstlisting}

\section{Testy wydajnościowe}

Do oceny wydajności aplikacji przeprowadzono podstawowe testy wydajnościowe, które polegały na symulacji dużego obciążenia generowanego przez jednoczesne zapytania użytkowników. Wyniki wykazały stabilność aplikacji pod obciążeniem oraz akceptowalne czasy odpowiedzi.

Testy przeprowadzono przy użyciu narzędzi takich jak Apache JMeter. Przykładowe parametry testu:

\begin{itemize}
    \item Liczba jednoczesnych użytkowników: 100.
    \item Czas trwania testu: 10 minut.
    \item Średni czas odpowiedzi aplikacji: poniżej 300ms.
\end{itemize}

Wnioski z testów pokazały, że aplikacja działa wydajnie i jest zdolna obsłużyć przewidywaną liczbę użytkowników.

\section{Podsumowanie wyników testów}

Przeprowadzone testy jednostkowe i integracyjne potwierdziły poprawność działania aplikacji oraz jej zgodność z założonymi wymaganiami funkcjonalnymi. Testy wydajnościowe dodatkowo potwierdziły stabilność aplikacji przy dużym obciążeniu, co umożliwia jej efektywne wdrożenie w środowisku edukacyjnym.
 % chapter4.tex zawiera treść rozdziału 4

%----------------------------------------------------------------------------------------
%	ROZDZIAŁ 5
%----------------------------------------------------------------------------------------
\chapter{Prezentacja działania aplikacji} W tym rozdziale zaprezentowano praktyczne działanie stworzonej aplikacji edukacyjnej z perspektywy użytkownika. Omówiono interfejs użytkownika oraz realizację kluczowych scenariuszy użycia systemu, takich jak rejestracja i logowanie, zapisywanie się na zajęcia, publikowanie materiałów, przesyłanie rozwiązań zadań czy wystawianie ocen i komentarzy. Ponadto przedstawiono funkcje administracyjne, w tym przegląd statystyk działania aplikacji. \section{Interfejs użytkownika} Interfejs użytkownika aplikacji został zaprojektowany z myślą o przejrzystości i intuicyjności. Do interfejsu wykorzystano bibliotekę Bootstrap, najpopularniejszy framework front-endowy (framework CSS/JS), który zapewnia responsywność i nowoczesny wygląd stron przy jednoczesnym ograniczeniu konieczności pisania dużej ilości własnego CSSu. Widoki HTML są generowane po stronie serwera za pomocą silnika szablonów Thymeleaf. Pozwala to łatwo wstrzykiwać dane serwera do widoków i wiązać interfejs użytkownika z logiką backendu \cite{thymeleaf-docs}. Najważniejsze widoki aplikacji to: \begin{itemize}
\item \textbf{Strona główna} – punkt startowy dla wszystkich użytkowników. Zawiera listę dostępnych kursów i najnowszych materiałów edukacyjnych opublikowanych na platformie. Na górze strony umieszczono pole wyszukiwania umożliwiające odnalezienie materiałów po słowach kluczowych (np. tytule lub opisie). Widoczne są również przyciski nawigacyjne do logowania i rejestracji (dla nowych użytkowników). Każdy kurs wyświetlany jest jako kafelek z nazwą przedmiotu i krótkim opisem, kliknięcie przenosi do szczegółów kursu.
\item \textbf{Widok kursu i lekcji} – po wybraniu konkretnego kursu (przedmiotu) student może zobaczyć jego szczegóły: listę lekcji (tematów) dostępnych w ramach kursu, informacje o prowadzącym (nauczycielu) oraz przycisk do zapisania się na kurs, jeśli użytkownik jeszcze nie jest uczestnikiem. Lekcje w ramach kursu są wyświetlane jako lista użytkownik może wybrać lekcję, aby zapoznać się z jej treścią.
\item \textbf{Widok materiału edukacyjnego (lekcji lub postu)} – prezentuje pełną treść materiału, np. zawartość lekcji lub posta dodanego przez nauczyciela. Oprócz treści (tekst, ewentualnie dołączony plik do pobrania, np. prezentacja lub PDF), widok ten zawiera sekcję komentarzy użytkowników oraz ocenę materiału. Studenci mogą dodawać komentarze, a zarówno studenci, jak i nauczyciele mogą oceniać materiał w skali 1-10.
\item \textbf{Panel użytkownika (Student)} – po zalogowaniu student otrzymuje dostęp do swojego panelu, gdzie może zarządzać uczestnictwem w kursach. W panelu użytkownika wyświetlane są m.in. lista kursów, na które użytkownik się zapisał (wraz z informacją o statusie zapisu oczekujący lub zaakceptowany), a także sekcja do przeglądania ocen otrzymanych za rozwiązania zadań.
Użytkownik może również w panelu zmienić swoje dane podane przy rejestracji.
\item \textbf{Panel nauczyciela} – nauczyciel po zalogowaniu widzi dedykowany panel z funkcjami dla prowadzących. Może tam zarządzać przydzielonymi mu kursami (przegląda listę swoich klas/kursów), dodawać nowe lekcje i zadania, a także przeglądać zadania przesłane przez studentów. W panelu nauczyciela znajduje się np. widok listy nadesłanych rozwiązań zadań, które oczekują na ocenę.
\item \textbf{Panel administracyjny} – dostępny wyłącznie dla użytkownika z rolą Administrator. Umożliwia on globalne zarządzanie systemem: listą wszystkich użytkowników (np. nadawanie roli nauczyciela wybranym użytkownikom, blokowanie użytkowników, itp.), listą wszystkich kursów (dodawanie lub usuwanie kursów, przypisywanie nauczycieli do kursów), a także przegląd opublikowanych materiałów (postów) w całym serwisie. Dodatkowo panel admina zawiera sekcję statystyk prezentującą dane o aktywności na platformie.
\end{itemize} Całość interfejsu jest spójna i utrzymana w ciemnym kolorze(tzw. \textit{dark mode} domyślnie włączony, z możliwością przełączenia na tryb jasny). Dzięki zastosowaniu dobrze znanych bibliotek i wzorców projektowych (Bootstrap, układ kart i list, ikony z biblioteki FontAwesome itp.), użytkownicy mogą łatwo odnaleźć potrzebne funkcje \cite{bootstrap-docs}. \section{Rejestracja i logowanie użytkowników} Proces rejestracji nowego użytkownika jest prosty i intuicyjny. Po wybraniu opcji "Register" użytkownik zostaje przekierowany do formularza rejestracji. Formularz wymaga podania podstawowych danych: nazwy użytkownika (loginu), hasła (dwukrotnie w celu weryfikacji), adresu e-mail oraz opcjonalnie imienia i nazwiska. Wszystkie pola są walidowane – zarówno po stronie przeglądarki (np. sprawdzenie minimalnej długości hasła, poprawności formatu email), jak i po stronie serwera (sprawdzenie unikalności nazwy użytkownika i adresu email w bazie). W przypadku niespełnienia wymagań walidowanych pól, użytkownik jest informowany odpowiednim komunikatem (np. "Nazwa użytkownika jest już zajęta" lub "Adres email jest już używany"). Po pomyślnym wypełnieniu formularza i jego wysłaniu, nowy użytkownik zostaje zapisany w systemie. Hasło użytkownika jest przechowywane w postaci haszu (z użyciem algorytmu BCrypt), co jest praktyką zalecaną przez Spring Security \cite{spring-security-in-action-2e}
. Dzięki jednokierunkowemu haszowaniu hasła nie są zapisywane jawnie i administrator nie ma dostępu do oryginalnego hasła. Rejestracja kończy się przekierowaniem do strony logowania z komunikatem o powodzeniu procesu (np. "Konto zostało utworzone."). Logowanie odbywa się poprzez dedykowaną stronę \texttt{/login}, gdzie użytkownik podaje swój login oraz hasło. Po poprawnym zalogowaniu (weryfikowanym przez Spring Security w oparciu o dane zapisane w Hazelcast) następuje przekierowanie na stronę główną z informacją powitalną. Od tego momentu w nawigacji widoczne są dodatkowe opcje w zależności od roli  np. dla nauczyciela link do panelu nauczyciela, dla admina link do panelu administracyjnego. Moduł logowania wykorzystuje Spring Security i wdraża rekomendowane zabezpieczenia, np. ograniczenie liczby prób logowania, aby utrudnić ataki typu brute-force \cite{spring-security-in-action-2e}, oraz unieważnianie sesji przy wylogowaniu i po dłuższej bezczynności. Dzięki temu system zabezpieczeń jest zgodny z ogólnie przyjętymi standardami. \section{Publikowanie i zarządzanie materiałami edukacyjnymi} Po zalogowaniu użytkownicy z odpowiednimi rolami mogą tworzyć i zarządzać zawartością edukacyjną na platformie. Przykładowo, nauczyciel (lub administrator) może dodawać nowe posty edukacyjne lub lekcje w ramach swoich kursów. Proces ten zazwyczaj przebiega według schematu: \begin{itemize}
\item Wybranie opcji \textbf{"Add post"} – w panelu nauczyciela dostępna jest zakładka umożliwiająca utworzenie nowego posta (materiału edukacyjnego) bądź nowej lekcji w ramach kursu.
\item \textbf{Wypełnienie formularza dodawania materiału} – nauczyciel wpisuje tytuł materiału, treść (np. opis lekcji lub artykuł edukacyjny). Formularz pozwala także na dołączenie pliku (np. dodatkowych materiałów PDF, prezentacji, arkuszy ćwiczeń). W projekcie obsługiwane jest dodawanie jednego pliku do materiału – plik ten jest zapisywany w systemie (w pamięci jako tablica bajtów) i może być później pobrany przez studentów.
\item \textbf{Zapisanie materiału} – po zatwierdzeniu formularza materiał zostaje zapisany poprzez odpowiedni kontroler (np. \texttt{AdminPostController} lub \texttt{TeacherPostController}) do repozytorium (\texttt{PostRepository} lub \texttt{LessonRepository}). Nowo utworzony obiekt otrzymuje unikalne ID, a ewentualny plik jest składowany we właściwościach obiektu (np. \texttt{attachment} i \texttt{attachmentFilename} w przypadku postu).
\item \textbf{Publikacja} – zapisany materiał staje się od razu dostępny na platformie. Posty mogą pojawić się na liście wszystkich materiałów widocznych na stronie głównej lub w odpowiedniej sekcji, zaś lekcja będzie widoczna w ramach danego kursu (dla studentów zapisanych na ten kurs lub dla wszystkich, jeśli kurs jest otwarty).
\item \textbf{Edycja i usuwanie} – nauczyciel lub administrator może w każdej chwili edytować zawartość istniejącego materiału (zmienić treść, podmienić załącznik) lub go usunąć, jeśli uzna, że jest nieaktualny lub niepoprawny. W aplikacji zaimplementowano widoki listy materiałów z akcjami "Edytuj" i "Usuń" przy każdym wpisie, dostępne dla uprawnionych użytkowników.
\end{itemize} Dzięki tym funkcjonalnościom platforma umożliwia dynamiczne dodawanie treści edukacyjnych przez kadrę nauczycielską. Studenci mają stały dostęp do aktualizowanych materiałów. Dodatkowo możliwość dołączenia plików czyni platformę elastyczną może służyć nie tylko do prezentowania tekstów, ale także udostępniania zadań domowych, artykułów naukowych, nagrań video lub innych zasobów. Ważnym elementem interaktywności jest system komentarzy i ocen. Każdy opublikowany materiał (np. post edukacyjny czy lekcja) może być oceniony oraz skomentowany przez społeczność użytkowników. Studenci mogą zadawać pytania lub dzielić się refleksjami w sekcji komentarzy pod materiałem, podczas gdy inni studenci i nauczyciele mogą te komentarze lajkować lub odpowiadać na nie. Oceny materiałów (realizowane w projekcie np. przez mechanizm \texttt{PostRating} lub \texttt{LessonRating}) pozwalają wyróżnić szczególnie wartościowe treści – średnia ocena lub liczba polubień może być wyświetlana obok tytułu materiału, dając nowym użytkownikom wskazówkę co do jakości. System ten zwiększa interaktywność platformy oraz motywuje autorów materiałów do utrzymania wysokiego poziomu merytorycznego. \section{Przeglądanie statystyk i funkcje administracyjne} Aplikacja udostępnia panel administracyjny zawierający najważniejsze statystyki dotyczące aktywności użytkowników i zawartości platformy. Panel statystyk jest dostępny tylko dla administratorów i stanowi pomocne narzędzie do monitorowania rozwoju aplikacji. Dane statystyczne są automatycznie aktualizowane w tle (zgodnie z opisem w poprzednim rozdziale) i prezentowane w przyjaznej formie graficznej. W panelu administracyjnym, na stronie głównej wyświetlane są m.in.: \begin{itemize}
\item \textbf{Liczba zarejestrowanych użytkowników} – całkowita liczba kont użytkowników w systemie, z podziałem na role (np. ile z nich to nauczyciele, a ile studenci).
\item \textbf{Liczba opublikowanych materiałów} – suma wszystkich postów edukacyjnych dodanych przez nauczycieli. Wskaźnik ten obrazuje skalę treści dostępnych na platformie.
\item \textbf{Liczba komentarzy} – łączna liczba komentarzy dodanych pod wszystkimi materiałami. Wysoka liczba komentarzy może świadczyć o dużym zaangażowaniu społeczności w dyskusje.
\item \textbf{Liczba zgłoszeń/zapisów} – administrator może również monitorować, ile zgłoszeń do kursów zostało złożonych (i ile oczekuje na akceptację). Ta statystyka pomaga śledzić popularność poszczególnych kursów.
\end{itemize} Do rysowania wykresów słupkowych w panelu użyto biblioteki do wykresów HTML5 Chart.js
, udostępnia gotowe typy wykresów (słupkowy, liniowy, kołowy itp.), co ułatwia wizualizację danych. Aplikacja przekazuje do widoku listy etykiet (np. \texttt{["postCount", "userCount", "teacherCount", "commentCount"]}) oraz odpowiadających im wartości, a skrypt \texttt{adminDashboard.js} renderuje na ich podstawie kolorowy wykres. Dzięki temu administrator otrzymuje przejrzysty obraz stanu systemu na pierwszy rzut oka może np. zauważyć dynamiczny przyrost liczby użytkowników czy też niewielką liczbę komentarzy (co może sugerować potrzebę zwiększenia interakcji). Poza statystykami, panel administracyjny zapewnia interfejs do zarządzania kluczowymi danymi:
\begin{itemize}
\item \textbf{Zarządzanie użytkownikami} – admin może przeglądać listę wszystkich kont, zmieniać role użytkowników (np. awansować studenta na nauczyciela jeśli zaistnieje taka potrzeba), a także usuwać konta naruszające regulamin.
\item \textbf{Zarządzanie kursami i klasami} – admin widzi listę wszystkich utworzonych kursów (klas). Może dodawać nowe kursy, przypisywać lub zmieniać nauczyciela prowadzącego dany kurs, a także usuwać kursy, które są nieaktywne. Co istotne, administrator ma podgląd na skład kursów  poprzez widok listy studentów zapisanych na dany przedmiot (z informacją o ich statusie: oczekujący bądź zaakceptowany). Jeśli kurs nie ma przypisanego nauczyciela, administrator może go wyznaczyć lub kurs będzie widoczny w panelu nauczyciela jako dostępny do przejęcia.
\item \textbf{Moderacja treści} – admin ma uprawnienia do przeglądu i edycji wszystkich postów i komentarzy. W razie konieczności może usunąć nieodpowiedni komentarz lub materiał.
\end{itemize} Przyjazny interfejs administracyjny, połączony z bogatym zestawem informacji statystycznych, pozwala na sprawne zarządzanie systemem i podejmowanie decyzji opartych na danych. Na przykład, jeśli administrator zauważy, że jakiś kurs cieszy się dużym zainteresowaniem (wiele zgłoszeń) ale nie ma przypisanego nauczyciela, może zareagować, przydzielając prowadzącego. Jeśli statystyki pokażą niski poziom aktywności (np. mało nowych postów lub komentarzy), może to być sygnał do zachęcenia nauczycieli do częstszego publikowania materiałów lub zorganizowania aktywności angażujących studentów. \section{Przykładowe scenariusze użycia aplikacji} Poniżej przedstawiono kilka przykładowych scenariuszy ilustrujących, jak przebiegają typowe interakcje użytkowników z systemem w praktyce. \subsection{Scenariusz 1: Dodanie nowego materiału przez nauczyciela} Nauczyciel loguje się do systemu i przechodzi do swojego panelu. Wybiera opcję \textit{"Dodaj nowy materiał"}. Otwiera się formularz, w którym nauczyciel podaje tytuł posta (np. "Wprowadzenie do Programowania"), wpisuje treść (kilka akapitów teorii, wprowadzenie do tematu zajęć) oraz dodaje plik PDF z prezentacją z wykładu. Po kliknięciu \textit{"Zapisz"}, materiał zostaje utworzony i zapisany. Nauczyciel zostaje przeniesiony na stronę listy swoich postów, gdzie widzi nowo dodany materiał na liście. Od tej chwili post jest widoczny dla wszystkich studentów na stronie głównej w sekcji aktualności lub w ramach przypisanego kursu (jeśli materiał jest częścią konkretnego kursu). Studenci mogą zapoznać się z treścią posta oraz pobrać załączoną prezentację. \subsection{Scenariusz 2: Ocena i komentarz do materiału przez studenta} Student przegląda listę dostępnych materiałów i otwiera interesujący go post (np. wspomniany "Wprowadzenie do Programowania"). Po przeczytaniu treści, student postanawia zostawić swoją opinię. W polu komentarza wpisuje pytanie do nauczyciela dotyczące niejasnej części materiału i publikuje komentarz. Komentarz natychmiast pojawia się pod postem, widoczny dla innych. Następnie student wystawia ocenę np. daje 9/10 punktów, sygnalizując, że materiał był bardzo pomocny. System zapisuje ocenę. Inni studenci, widząc komentarz, mogą go polubić jeśli również mieli podobne pytanie. Nauczyciel, gdy zaloguje się następnym razem, zobaczy powiadomienie o nowym komentarzu do swojego posta. Może odpowiedzieć studentowi, wyjaśniając wątpliwości, co również czyni poprzez sekcję komentarzy. Dzięki temu platforma wspiera interakcję między studentami a nauczycielami, czyniąc proces nauki bardziej interaktywnym. \subsection{Scenariusz 3: Rejestracja nowego użytkownika i zapis na kurs} Nowy użytkownik (student) odwiedza stronę główną aplikacji i chce uzyskać dostęp do materiałów, które oferuje platforma. Wybiera opcję \textit{"Zarejestruj się"} i wypełnia formularz rejestracyjny podając unikalny login, hasło oraz adres email. Po wysłaniu formularza otrzymuje komunikat o pomyślnej rejestracji. Następnie loguje się za pomocą utworzonego przed chwilą konta. Po zalogowaniu, jako nowy użytkownik (domyślnie student), widzi stronę główną z listą kursów. Wybiera kurs, który go interesuje – np. "Programowanie w Javie". W widoku szczegółów kursu klika przycisk \textit{"Zapisz się na kurs"}. System rejestruje jego zgłoszenie: jeśli kurs jest otwarty (automatyczne zatwierdzanie zapisów jest włączone), od razu uzyskuje status uczestnika i może przeglądać wszystkie lekcje tego kursu. Jeśli kurs wymaga akceptacji (autozatwierdzanie wyłączone), zgłoszenie otrzymuje status "PENDING" (oczekujące). Student widzi informację, że jego prośba o dołączenie czeka na zatwierdzenie przez prowadzącego lub administratora. Może kontynuować korzystanie z innych dostępnych materiałów na platformie. Gdy nauczyciel przydzielony do kursu zaakceptuje zgłoszenie, student zostanie powiadomiony przy kolejnym odwiedzeniu panelu użytkownika zobaczy, że jego status w kursie "Programowanie w Javie" zmienił się na aktywny i od tej chwili ma pełny dostęp do zawartości kursu. \subsection{Scenariusz 4: Przesłanie zadania przez studenta i ocena przez nauczyciela} W trakcie trwania kursu nauczyciel dodaje do jednej z lekcji zadanie domowe (np. projekt do wykonania lub zestaw zadań do rozwiązania). Student, który jest uczestnikiem kursu, po zapoznaniu się z treścią zadania przygotowuje rozwiązanie na swoim komputerze (np. w formie pliku z kodem źródłowym lub dokumentu PDF z odpowiedziami). Następnie loguje się do platformy, przechodzi do widoku szczegółów lekcji, gdzie dostępny jest formularz \textit{"Prześlij rozwiązanie"}. Student wybiera plik ze swojego komputera i zatwierdza przesłanie. Aplikacja przyjmuje plik, zapisuje go w repozytorium \texttt{TaskSubmissionRepository} wraz z informacją, który student i do którego zadania go przesłał. Status tego zgłoszenia jest początkowo "nieoceniony". Nauczyciel, chcąc sprawdzić nadesłane prace, loguje się i otwiera swój panel nauczyciela. W sekcji \textit{"Przesłane rozwiązania"} widzi listę wszystkich nadesłanych przez studentów plików do zadań, które nie zostały jeszcze ocenione. Dla każdego wpisu widnieje nazwa studenta, nazwa zadania oraz opcja pobrania pliku. Nauczyciel klika \textit{"Pobierz"} – aplikacja, dzięki \texttt{FileDownloadController}, zwraca plik wysłany przez studenta (np. kod źródłowy lub dokument) jako załącznik do pobrania. Nauczyciel ocenia pracę offline, po czym wraca do platformy. Przy każdym rozwiązaniu w panelu dostępny jest formularz \textit{"Wystaw ocenę"} nauczyciel wpisuje punktację (w skali od 0 do 10) oraz opcjonalny komentarz zwrotny dla studenta. Po wysłaniu formularza oceny, system zapisuje ocenę i komentarz w obiekcie \texttt{TaskSubmission} (pola \texttt{grade} oraz \texttt{teacherComments}). Rozwiązanie znika z listy "oczekujących na ocenę", stając się ocenionym. Student, przechodząc do swojego panelu, może teraz zobaczyć wynik  przy danym zadaniu pojawia się ocena wystawiona przez nauczyciela oraz ewentualne uwagi. Dzięki temu ma informację zwrotną, czy dobrze wykonał pracę, i czego ewentualnie powinien się nauczyć. Opisane powyżej scenariusze ilustrują kluczowe funkcjonalności aplikacji i sposób, w jaki różne role użytkowników (student, nauczyciel, administrator) współdziałają ze sobą za pośrednictwem systemu. Platforma jest intuicyjna w obsłudze, nawet nowy użytkownik jest w stanie szybko zarejestrować się, zapisać na kurs i rozpocząć naukę korzystając z dostępnych materiałów. Jednocześnie nauczyciele mają do dyspozycji narzędzia do efektywnego przekazywania wiedzy (publikacje materiałów, zadawanie prac domowych, ocenianie) bez konieczności użycia zewnętrznych kanałów komunikacji. Wreszcie, funkcje administracyjne gwarantują, że całość może być nadzorowana i dostosowywana przez uprawnione osoby, co zapewnia stabilność i wysoką jakość działania aplikacji. % chapter5.tex zawiera treść rozdziału 5

%----------------------------------------------------------------------------------------
%	ROZDZIAŁ 6
%----------------------------------------------------------------------------------------
\chapter{Testowanie i ocena aplikacji} W niniejszym rozdziale przedstawiono proces testowania aplikacji oraz ocenę poprawności jej działania pod względem funkcjonalnym i niefunkcjonalnym. Omówione zostały podejścia wykorzystane w testach jednostkowych, integracyjnych, testy wydajnościowe oraz testy end-to-end. Celem testowania było upewnienie się, że wszystkie warstwy systemu współpracują prawidłowo, a aplikacja spełnia postawione wymagania w warunkach zbliżonych do rzeczywistych obciążeń. \section{Wprowadzenie do testowania aplikacji} Testowanie stanowi kluczowy element procesu tworzenia oprogramowania, umożliwiający wykrycie błędów oraz weryfikację poprawności zaimplementowanych funkcjonalności na wczesnym etapie. W ramach tego projektu zastosowano różne poziomy testów, aby uzyskać pewność co do jakości kodu: \begin{itemize}
\item \textbf{Testy jednostkowe} –Testy jednostkowe sprawdzają pojedyncze jednostki kodu. Pozwalają sprawdzić logikę wybranego fragmentu kodu w oderwaniu od kontekstu aplikacji (np. metodę serwisu aktualizującą statystyki lub walidującą dane wejściowe) \cite{effective-unit-testing}.
\item \textbf{Testy integracyjne} – sprawdzają, czy współpraca wielu komponentów (warstwa web, logika, dostęp do danych, mechanizmy bezpieczeństwa) przebiega poprawnie w uruchomionym kontekście aplikacji. \cite{spring-docs}
\item \textbf{Testy wydajnościowe} – oceniają zachowanie aplikacji przy wzrastającym obciążeniu i równoległych użytkownikach, aby upewnić się, że czasy odpowiedzi mieszczą się w akceptowalnych granicach \cite{jmeter-docs}.
\end{itemize} Tak wielopoziomowe podejście do testowania umożliwia wykrycie różnych klas błędów od drobnych pomyłek w logice pojedynczej metody, przez problemy z konfiguracją kontekstu Spring, aż po ewentualne wąskie gardła w wydajności. \section{Testy jednostkowe} Testy jednostkowe zostały napisane z użyciem frameworka JUnit 5, który jest obecnie standardem w ekosystemie Java. Do izolowania zależności wykorzystano bibliotekę Mockito, pozwalającą na tworzenie próbnych obiektów (moków) zastępujących np. warstwę repozytorium podczas testowania serwisu, lub warstwę serwisu podczas testowania kontrolera \cite{mockito-docs}. Dzięki temu można skupić się na logice danej jednostki kodu, nie przejmując się działaniem pozostałych komponentów (które są zasymulowane). Przykładowo, testy jednostkowe objęły klasy:
\begin{itemize}
\item \textbf{Serwisy} – np. \texttt{StatsService}, gdzie sprawdzono czy metoda aktualizująca statystyki poprawnie odczytuje dane z repozytoriów i zapisuje zaktualizowane statystyki. Zastosowano tutaj \textit{stub} na metodach repozytoriów \cite{mockito-docs, junit-docs}(\texttt{postRepository}, \texttt{userRepository}, itp.) zwracające z góry ustalone wartości, by zasymulować określony stan systemu, a następnie weryfikowano, czy serwis wywołuje metodę zapisu statystyk dokładnie raz z poprawnymi parametrami.
\item \textbf{Kontrolery} – testowane zarówno metodami bezpośredniego wywołania, jak i z użyciem Spring MockMVC. W pierwszym podejściu tworzono instancję kontrolera i za pomocą moków podstawiano zależności (np. \texttt{UserRepository}, \texttt{PasswordEncoder}), a następnie wywoływano metodę kontrolera tak, jakby obsługiwała żądanie. Sprawdzano zwracany widok oraz efekty uboczne (np. czy nowy użytkownik został zapisany w repozytorium po rejestracji). W drugim podejściu (MockMVC) uruchamiano wbudowany serwer Tomcat w trybie testowym i wykonywano sztuczne zapytania HTTP do endpointów, sprawdzając kod odpowiedzi i fragmenty wygenerowanej strony. \cite{spring-docs}
\item \textbf{Repozytoria} – ponieważ repozytoria Hazelcast nie korzystają z klasycznego mechanizmu ORM ani bazy SQL, testy jednostkowe repozytoriów polegały głównie na sprawdzeniu logiki metod filtrujących. Np. \texttt{ClassSignUpRepository.findBySchoolClassIdAndUserId} filtruje kolekcję zapisów, więc utworzono kilka obiektów \texttt{ClassSignUp} w kontrolowanej mapie i upewniono się, że metoda znajduje właściwy obiekt lub zwraca \texttt{Optional.empty()} gdy brak dopasowania \cite{hazelcast-docs}.
\end{itemize}
Poniżej przedstawiono fragment przykładowego testu jednostkowego dla serwisu statystyk \texttt{StatsService}, demonstrującego użycie Mockito do symulacji repozytoriów i weryfikacji zachowania:
\begin{lstlisting}[language=Java,
  caption={test kodu aktualizowania statystyk. Źródło: opracowanie własne},
  label={lst:testStatisticUpdater},
  captionpos=b]
@Test
void shouldUpdateStatisticsCorrectly() {
// Symulacja wartości zwracanych przez repozytoria:
when(postRepository.count()).thenReturn(10L);
when(userRepository.countByRole("TEACHER")).thenReturn(2L);
when(userRepository.countByRole("USER")).thenReturn(5L);
when(commentRepository.count()).thenReturn(20L);
// Wywołanie testowanej metody:
statsService.updateStatistics();
// Weryfikacja, że statystyki zostały zapisane (4 statystyki do zapisania):
verify(statsRepository, times(4)).save(any(AppStatistic.class));
}
\end{lstlisting} W powyższym teście założono, że w systemie jest 10 postów, 2 nauczycieli, 5 zwykłych użytkowników i 20 komentarzy. Po wywołaniu \texttt{updateStatistics()} oczekujemy, że serwis spróbuje zapisać każdą z czterech statystyk (postCount, teacherCount, userCount, commentCount) dokładnie raz – stąd \texttt{verify} z \texttt{times(4)} na \texttt{statsRepository.save(...)}. Wykorzystanie mechanizmów Mockito pozwoliło na izolację testowanej metody od faktycznej bazy danych i innych serwisów  test jest szybki i deterministyczny. \section{Testy integracyjne} Testy integracyjne przeprowadzono z wykorzystaniem wbudowanych możliwości Spring Boot Test. Uruchamiając kontekst całej aplikacji w trybie testowym, można symulować rzeczywiste scenariusze użytkownika, sprawdzając czy wszystkie warstwy (od kontrolera, przez serwisy, po repozytoria i magazyn danych) działają wspólnie poprawnie. W projekcie wykorzystano głównie podejście \textbf{end-to-end}, gdzie testy integracyjne zachowywały się jak klienci aplikacji korzystający z niej przez protokół HTTP. Do realizacji takich testów zastosowano:
\begin{itemize}
  \item \texttt{SpringBootTest} z \texttt{webEnvironment=RANDOM\_PORT} --
dzięki \texttt{RANDOM\_PORT} testy uruchamiają aplikację na wolnym porcie, co pozwala na rzeczywisty dostęp HTTP do endpointów.
  \item \texttt{TestRestTemplate} --  klient HTTP z pakietu Spring Boot Test, umożliwiający wykonywanie żądań HTTP do uruchomionej aplikacji.
  \item profil \texttt{test} -- w testach aktywowano profil testowy, który
    używa osobnej konfiguracji Hazelcast (mniejsza liczba danych
    inicjalnych, wyłączone zadania cykliczne poprzez
    \texttt{spring.task.scheduling.enabled=false}, aby testy były
    deterministyczne).
\end{itemize}
 Przykładem testu integracyjnego może być sprawdzenie działania strony głównej aplikacji: \begin{lstlisting}[language=Java,
  caption={integracyjny test. Źródło: opracowanie własne},
  label={lst:integrationTest},
  captionpos=b]
@SpringBootTest(webEnvironment = SpringBootTest.WebEnvironment.RANDOM_PORT)
@ActiveProfiles("test")
class ApplicationIntegrationTests {
@Autowired
private TestRestTemplate restTemplate;
@Test
void homePageShouldLoadSuccessfully() {
ResponseEntity<String> response = restTemplate.getForEntity("/", String.class);
assertEquals(HttpStatus.OK, response.getStatusCode());
assertTrue(response.getBody().contains("Welcome to MyEduShare"));
}
}
\end{lstlisting} 
W powyższym teście uruchamiamy aplikację, wykonujemy zapytanie GET na
\texttt{"/"} (strona główna) i sprawdzamy, czy odpowiedź ma kod 200OK oraz
czy zwrócona treść HTML zawiera oczekiwany fragment (np. tytuł lub
charakterystyczny tekst powitalny strony). Taki test potwierdza, że
podstawowa ścieżka (wejście na stronę główną) działa prawidłowo-- aplikacja
się uruchamia, kontroler główny zwraca widok, a mechanizmy szablonów
poprawnie generują stronę.

Bardziej złożone testy integracyjne zostały stworzone dla typowych
scenariuszy użycia, opisanych w poprzednim rozdziale. Na przykład
przygotowano test end-to-end symulujący cały przepływ:
\textit{nauczyciel tworzy nowy kurs i zadanie $\to$ student rejestruje się i
zapisuje na ten kurs $\to$ student przesyła rozwiązanie zadania $\to$
nauczyciel pobiera i ocenia rozwiązanie $\to$ sprawdzenie, że ocena jest
zapisana}. Tego typu test (zaimplementowany w klasie
\texttt{TeacherStudentFlowE2eTest}) wykorzystuje \texttt{TestRestTemplate}
do wykonywania kolejnych żądań POST/GET, naśladując akcje formularzy (np.
przesyłając dane rejestracji w żądaniu POST do
\texttt{/perform\_register}, logując różne konta poprzez
\texttt{TestRestTemplate} z odpowiednimi ciasteczkami sesyjnymi itp.) \cite{spring-docs}. Po
serii akcji test weryfikuje stan bazy-- np. sprawdza, czy w repozytorium
\texttt{TaskSubmissionRepository} pojawił się wpis z oceną nauczyciela. Takie
kompleksowe testy dają dużą pewność, że kluczowe funkcjonalności aplikacji
działają poprawnie w całym przekroju systemu.

W trakcie testów integracyjnych aplikacja korzystała z tej samej bazy
Hazelcast (tyle że w trybie „test”, z osobną konfiguracją) co w normalnym
działaniu. Oznacza to, że wszystkie
operacje na danych w testach były wykonywane również w pamięci (co
zapewniło szybkość testów i identyczne warunki działania jak w
środowisku produkcyjnym aplikacji). Podejście to upraszcza testy-- brak
translacji do innej technologii (SQL) oznacza, że testujemy dokładnie te
same ścieżki kodu co podczas realnego działania aplikacji \cite{hazelcast-docs}. Testy
integracyjne potwierdziły poprawne współdziałanie komponentów aplikacji:
wszystkie zaplanowane scenariusze (rejestracja, logowanie, dodawanie
postów, zapisy na kurs, przesyłanie zadań, oceny itp.) zakończyły się
wynikiem zgodnym z oczekiwaniami. Również obsługa błędnych ścieżek (np.
próba dostępu do zasobów bez autoryzacji, podwójna rejestracja z tym samym
e-mailem) została zweryfikowana-- aplikacja w takich przypadkach zwraca komunikaty lub kody błędów (kod 403) \cite{spring-docs}.

\section{Testy wydajnościowe}

Ostatnim etapem było zbadanie zachowania aplikacji pod obciążeniem. W tym celu przygotowano plan w Apache JMeter, który symuluje jednoczesne działania wielu użytkowników i pozwala mierzyć czasy odpowiedzi oraz stabilność przepływów \cite{jmeter-docs}. Skrypt odwzorowuje typowe ścieżki użytkowników aplikacji. Poniżej zwięzły opis kluczowych elementów planu testowego.

\begin{itemize}
  \item \textbf{HTTP Request Defaults} ---  konfiguracja hosta (np. \texttt{localhost:8080}),
        protokołu i bazowej ścieżki, ułatwia utrzymanie skryptu \cite{jmeter-docs}.
  \item \textbf{HTTP Cookie \& Cache Manager} --- podtrzymanie sesji zalogowanego użytkownika i obsługa ciasteczek oraz cache przeglądarki \cite{jmeter-docs}.
  \item \textbf{CSV Data Set Config} --- strzykiwanie danych wejściowych (loginy/hasła i identyfikatory zasobów): \texttt{users\_teachers.csv}, \texttt{post\_ids.csv}, \texttt{class\_ids.csv} itp. \cite{jmeter-docs}.
  \item \textbf{Pobranie i przekazanie CSRF} --- po wejściu na \texttt{/login} token CSRF jest wyekstrahowany z HTML i dołączany do żądań \texttt{POST} (nagłówek \texttt{X-CSRF-TOKEN} i/lub pole formularza), zgodnie z mechanizmem ochrony w Spring Security \cite{spring-docs}.
  \item \textbf{Korelacja identyfikatorów} — ID pobierane z plików CSV lub z odpowiedzi (HTML/JSON) i przekazywane do kolejnych kroków \cite{jmeter-docs}.
  \item \textbf{Korelacja identyfikatorów} ---  ID pobierane z plików CSV lub z odpowiedzi (HTML/JSON) i przekazywane do kolejnych kroków \cite{jmeter-docs}.
  \item \textbf{Przykładowe ścieżki API użyte podczas testów}:
    \begin{itemize}
      \item \emph{Logowanie:} \texttt{GET /login} $\rightarrow$ \texttt{POST /perform\_login}
            z tokenem CSRF i danymi z CSV.
      \item \emph{Dodanie posta (nauczyciel):} \texttt{POST /teacher/posts/new}
      \item \emph{Ocena posta (użytkownik):} \texttt{POST /posts/\{postId\}/rate}
    \end{itemize}
  \item \textbf{liczniki czasu} --- \emph{Uniform Random Timer} (np. 500--2000 ms) między akcjami,
        aby zasymulować naturalne tempo klikania \cite{jmeter-docs}.
  \item \textbf{Kontrola tempa} --- opcjonalny \emph{Constant Throughput Timer} oraz/lub
        \emph{Throughput Controller}, aby utrzymać docelową liczbę żądań \cite{jmeter-docs}.
  \item \textbf{Assercje} --- weryfikacja kodów odpowiedzi (200/302), obecności oczekiwanych elementów w plikach HTML/JSON,
        co potwierdza poprawność funkcjonalną ścieżek API.
  \item \textbf{Zapisy wyników} --- \emph{Simple Data Writer} do JTL (pełne dane pomiarowe) oraz raporty HTML \cite{jmeter-docs}.
\end{itemize}

Fragment struktury planu:

\begin{lstlisting}[caption={Fragment planu testów JMeter. Źródło: opracowanie własne},
  label={lst:jmeter-plan}, captionpos=b]
Test Plan
 +-- HTTP Request Defaults (http://localhost:8080)
 +-- HTTP Cookie Manager, HTTP Cache Manager
 +-- CSV Data Set: users_teachers.csv, post_ids.csv, class_ids.csv
 +-- Thread Group (VUs, ramp-up, duration)
 |   +-- GET /login  -> extract CSRF
 |   +-- POST /perform_login  (CSRF, credentials from CSV)
 |   +-- POST /teacher/posts/new  (CSRF, param title/content)
 |   +-- POST /teacher/classes/${CLASS_ID}/lessons/new  (CSRF)
 |   +-- POST /posts/${POST_ID}/comments  (CSRF, param text)
 |   +-- POST /posts/${POST_ID}/rate  (CSRF, param rating)
 |   +-- Simple Data Writer (results.jtl), HTML Report
\end{lstlisting}


\subsection{Parametry przykładowego testu wydajnościowego}

Poniżej zestawiono parametry biegu oraz krótkie uzasadnienie doboru wartości.

\begin{itemize}
  \item \textbf{Liczba równoczesnych użytkowników (VUs):}
        100, 250, 600, 1000 \;-- cztery poziomy obciążenia pozwalające ocenić trend skalowania.
  \item \textbf{Ramp-up:} 60~s (dla 100 VUs), 120~s (dla 250 VUs), 180~s (dla 600 VUs), 240~s (dla 1000 VUs);
        płynne narastanie wątków ogranicza skokowe zmiany i stabilizuje pomiary \cite{jmeter-docs}.
  \item \textbf{Czas trwania:} 10~min ciągłego wykonywania akcji (po ramp-up), co daje reprezentatywny przekrój próbek \cite{jmeter-docs}.
  \item \textbf{Tempo akcji (liczniki czasu):} losowe odstępy 0.5--2.0~s między krokami, aby zbliżyć się do realnego użycia aplikacji \cite{jmeter-docs}.
\item \textbf{Zakres endpointów:}
    \begin{itemize}
        \item \texttt{/perform\_login}
        \item \texttt{/teacher/posts/new}
        \item \texttt{/teacher/classes/\{classId\}/lessons/new}
        \item \texttt{/posts/\{postId\}/comments}
        \item \texttt{/posts/\{postId\}/rate}
    \end{itemize}
  \item \textbf{Dane wejściowe:} loginy i hasła z \texttt{users\_teachers.csv}, identyfikatory klas i postów
        (\texttt{class\_ids.csv}, \texttt{post\_ids.csv}); wartości pól (tytuł/treść/komentarz/ocena) parametryzowane.
  \item \textbf{CSRF i sesja:} token CSRF automatycznie pobierany z widoku logowania i dołączany do wszystkich żądań POST
        sesja jest utrzymywana przez \emph{HTTP Cookie Manager} \cite{jmeter-docs}.
  \item \textbf{Zapisy metryk:} pełne zapisy próbek w pliku JTL, raporty HTML z wartościami
        \emph{Mediana, Średnia, p90, Min, Max}.
\end{itemize}

W tak zdefiniowanym scenariuszu skrypt mierzy rzeczywiste czasowe koszty przetwarzania operacji.
\subsection{Wyniki szczegółowe: metryki dla kolejnych poziomów VUs}

Poniżej przedstawiono metryki czasów odpowiedzi (w~ms) dla czterech kluczowych operacji:
\textbf{Post} (dodanie posta), \textbf{Komentarz} (dodanie komentarza), \textbf{Logowanie} oraz  \textbf{Lekcja} (dodanie lekcji).
Dla każdego poziomu obciążenia (100/250/600/1000 VUs) pięć metryk: \emph{Mediana},
\emph{Średnia}, \emph{p90}, \emph{Min} oraz \emph{Max}.

\FloatBarrier
\clearpage

% ===================== 100 VUs =====================
\subsubsection{100 VUs} \begin{figure}[H]\centering \includegraphics[width=\textwidth]{Dyplom-styl/chart_simple_100VU_mediana.png} \caption{100 VUs --- Mediana (ms). Źródło: Opracowanie własne}\label{fig:100-mediana} \end{figure} Mediany są niskie dla wszystkich operacji, operacje utrzymują komfortowy poziom opóźnień. \begin{figure}[H]\centering \includegraphics[width=\textwidth]{Dyplom-styl/chart_simple_100VU_srednia.png} \caption{100 VUs --- Średnia (ms). Źródło: Opracowanie własne}\label{fig:100-srednia} \end{figure} Średnie są zbliżone do median, co potwierdza stabilny profil obciążeń bez odchyleń. \begin{figure}[H]\centering \includegraphics[width=\textwidth]{Dyplom-styl/chart_simple_100VU_p90.png} \caption{100 VUs --- p90 (ms). Źródło: Opracowanie własne}\label{fig:100-p90} \end{figure} p90 utrzymuje bezpieczny dystans od median, rezerwa wydajności jest wyraźna. \begin{figure}[H]\centering \includegraphics[width=\textwidth]{Dyplom-styl/chart_100VU_min_clean.png} \caption{100 VUs --- Min (ms). Źródło: Opracowanie własne }\label{fig:100-min} \end{figure} Minimalne czasy potwierdzają niskie koszty, żądań i szybką obsługę w pamięci.
\begin{figure}[H]\centering \includegraphics[width=\textwidth]{Dyplom-styl/chart_simple_100VU_max.png} \caption{100 VUs --- Max (ms). Źródło: Opracowanie własne }\label{fig:100-min} \end{figure} Maksymalne wartości występują okazjonalnie nie wpływając na odczucie podczas używania aplikacji.
% ===================== 250 VUs =====================
\subsubsection{250 VUs}

\begin{figure}[H]\centering
\includegraphics[width=\textwidth]{Dyplom-styl/chart_250VU_mediana.png}
\caption{250 VUs --- Mediana (ms). Źródło: Opracowanie własne}\label{fig:250-mediana}
\end{figure}
Wzrost obciążenia skutkuje przewidywalnym, łagodnym wzrostem median, brak skoków.

\begin{figure}[H]\centering
\includegraphics[width=\textwidth]{Dyplom-styl/chart_250VU_srednia.png}
\caption{250 VUs --- Średnia (ms). Źródło: Opracowanie własne}\label{fig:250-srednia}
\end{figure}
Średnie pomiarów pozostają w okolicy median, co potwierdza równomierny rozkład opóźnień.

\begin{figure}[H]\centering
\includegraphics[width=\textwidth]{Dyplom-styl/chart_250VU_p90.png}
\caption{250 VUs --- p90 (ms). Źródło: Opracowanie własne}\label{fig:250-p90}
\end{figure}
p90 rośnie proporcjonalnie do obciążenia, utrzymując komfortowy zapas względem median.

\begin{figure}[H]\centering
\includegraphics[width=\textwidth]{Dyplom-styl/chart_250VU_min_clean.png}
\caption{250 VUs --- Min (ms). Źródło: Opracowanie własne}\label{fig:250-min}
\end{figure}
Minimalne czasy pozostają bardzo niskie.

\begin{figure}[H]\centering
\includegraphics[width=\textwidth]{Dyplom-styl/chart_250VU_max_all4.png}
\caption{250 VUs --- Max (ms). Źródło: Opracowanie własne}\label{fig:250-max}
\end{figure}
Okazjonalne wartości szczytowe są pojedyncze i nie determinują ogólnego odczucia szybkości.

% ===================== 600 VUs =====================
\subsubsection{600 VUs}

\begin{figure}[H]\centering
\includegraphics[width=\textwidth]{Dyplom-styl/chart_600VU_mediana.png}
\caption{600 VUs --- Mediana (ms). Źródło: Opracowanie własne}\label{fig:600-mediana}
\end{figure}
System skaluje się harmonijnie, mediany rosną łagodnie.

\begin{figure}[H]\centering
\includegraphics[width=\textwidth]{Dyplom-styl/chart_600VU_srednia.png}
\caption{600 VUs --- Średnia (ms). Źródło: Opracowanie własne}\label{fig:600-srednia}
\end{figure}
Średnie potwierdzają stabilność, opóźnienia pozostają przewidywalne.

\begin{figure}[H]\centering
\includegraphics[width=\textwidth]{Dyplom-styl/chart_600VU_p90.png}
\caption{600 VUs --- p90 (ms). Źródło: Opracowanie własne}\label{fig:600-p90}
\end{figure}
p90 utrzymuje klarowną separację od median, co zapewnia komfort użytkownika także przy wyższych obciążeniach.

\begin{figure}[H]\centering
\includegraphics[width=\textwidth]{Dyplom-styl/chart_600VU_min_clean.png}
\caption{600 VUs --- Min (ms). Źródło: Opracowanie własne}\label{fig:600-min}
\end{figure}
Najniższe czasy nie ulegają pogorszeniu.

\begin{figure}[H]\centering
\includegraphics[width=\textwidth]{Dyplom-styl/chart_600VU_max_all4.png}
\caption{600 VUs --- Max (ms). Źródło: Opracowanie własne}\label{fig:600-max}
\end{figure}
Maksymalne wartości nadal pozostają w granicach komfortowego użytkowania.

% ===================== 1000 VUs =====================
\subsubsection{1000 VUs}

\begin{figure}[H]\centering
\includegraphics[width=\textwidth]{Dyplom-styl/chart_1000VU_mediana.png}
\caption{1000 VUs --- Mediana (ms). Źródło: Opracowanie własne}\label{fig:1000-mediana}
\end{figure}
Przy 1000 VUs mediany są nadal niskie, aplikacja zachowuje responsywność i płynność pracy.

\begin{figure}[H]\centering
\includegraphics[width=\textwidth]{Dyplom-styl/chart_1000VU_srednia.png}
\caption{1000 VUs --- Średnia (ms). Źródło: Opracowanie własne}\label{fig:1000-srednia}
\end{figure}
Średnie czasy potwierdzają stabilność, widoczna jest spójność wyników.

\begin{figure}[H]\centering
\includegraphics[width=\textwidth]{Dyplom-styl/chart_1000VU_p90.png}
\caption{1000 VUs --- p90 (ms). Źródło: Opracowanie własne}\label{fig:1000-p90}
\end{figure}
p90 pozostaje w rozsądnym zakresie, co gwarantuje komfortowe użytkowanie.

\begin{figure}[H]\centering
\includegraphics[width=\textwidth]{Dyplom-styl/chart_1000VU_min_clean.png}
\caption{1000 VUs --- Min (ms). Źródło: Opracowanie własne}\label{fig:1000-min}
\end{figure}
Minimalne czasy są bardzo niskie, przepływ żądań/zapytań jest efektywny.

\begin{figure}[H]\centering
\includegraphics[width=\textwidth]{Dyplom-styl/chart_1000VU_max_all4.png}
\caption{1000 VUs --- Max (ms). Źródło: Opracowanie własne}\label{fig:1000-max}
\end{figure}
%\FloatBarrier 
Wartości maksymalne zachowują akceptowalny poziom, brak symptomów przeciążenia.
\subsection{podsumowanie wyników na wykresach liniowych}

\begin{figure}[H]\centering
\includegraphics[width=\textwidth]{Dyplom-styl/line_komentarz_summary.png}
\caption{komentarz - wykres liniowy podsumowujący wyniki. Źródło: Opracowanie własne}\label{fig:line-post-summary}
\end{figure}
Trendy rosną łagodnie wraz z liczbą VU, operacja pozostaje responsywna podczas trwania testów.

\begin{figure}[H]\centering
\includegraphics[width=\textwidth]{Dyplom-styl/line_lekcja_summary.png}
\caption{dodanie komentarza — wykres liniowy podsumowujący wyniki. Źródło: Opracowanie własne}\label{fig:line-komentarz-summary}
\end{figure}
Stabilny wzrost metryk wraz z VU bez skokowych zmian, p90 pozostaje pod kontrolą, a operacja skaluje się przewidywalnie.

\begin{figure}[H]\centering
\includegraphics[width=\textwidth]{Dyplom-styl/line_post_summary.png}
\caption{dodanie "post" — wykres liniowy podsumowujący wyniki. Źródło: Opracowanie własne}\label{fig:line-post-summary}
\end{figure}
Czasy rosną proporcjonalnie do VU, p90 pozostaje blisko mediany, co potwierdza stabilne zachowanie pod obciążeniem.

\begin{figure}[H]\centering
\includegraphics[width=\textwidth]{Dyplom-styl/line_logowanie_summary.png}
\caption{logowanie — wykres liniowy podsumowujący wyniki. Źródło: Opracowanie własne}\label{fig:line-logowanie-summary}
\end{figure}
Najkrótsze czasy w całej próbie, nawet przy 1000 VUs utrzymana niska mediana i p90, co zapewnia bardzo dobrą responsywność.
\FloatBarrier 

\subsection{Podsumowanie wyników w tabelach}

\begingroup
  \let\oldcaption\caption
  \renewcommand{\caption}[2][]{\oldcaption[#1]{#2. Źródło: opracowanie własne}}
  \begin{table}[H]
\centering
\caption{Mediana czasu odpowiedzi (ms) — wiersze: VUs, kolumny: endpoint}
\label{tab:mediana-vs-vus}
\begin{tabular}{rrrrr}
\toprule
 VUs &   Post &  Komentarz &  Logowanie &  Lekcja \\
\midrule
 100 & 1096.0 &      975.0 &       90.0 &   703.0 \\
 250 & 1745.5 &     1589.5 &      108.0 &  1245.0 \\
 600 & 1744.0 &     1997.0 &       90.0 &  1460.0 \\
1000 & 2415.0 &     2318.0 &      109.0 &  1944.0 \\
\bottomrule
\end{tabular}
\end{table}

\endgroup
Mediany rosną proporcjonalnie do VUs, z zachowaniem niskich wartości i czytelnej przewagi logowania.

\begingroup
  \let\oldcaption\caption
  \renewcommand{\caption}[2][]{\oldcaption[#1]{#2. Źródło: opracowanie własne}}
  \begin{table}[H]
\centering
\caption{Średni czas czasu odpowiedzi (ms) — wiersze: VUs, kolumny: endpoint}
\label{tab:srednia-vs-vus}
\begin{tabular}{rrrrr}
\toprule
 VUs &        Post &   Komentarz &  Logowanie &      Lekcja \\
\midrule
 100 & 1024.152679 &  959.206359 &   150.8700 &  740.664055 \\
 250 & 1680.865622 & 1562.238743 &   209.0150 & 1250.261864 \\
 600 & 1764.604294 & 1861.691186 &   209.7275 & 1459.649407 \\
1000 & 2315.134783 & 2243.617152 &   251.5425 & 1955.728077 \\
\bottomrule
\end{tabular}
\end{table}

\endgroup
Średnie pokrywają się z trendem median, wyniki są spójne między operacjami.

\begingroup
  \let\oldcaption\caption
  \renewcommand{\caption}[2][]{\oldcaption[#1]{#2. Źródło: opracowanie własne}}
  \begin{table}[H]
\centering
\caption{p90 czasu odpowiedzi (ms) — wiersze: VUs, kolumny: endpoint}
\label{tab:p90-vs-vus}
\begin{tabular}{rrrrr}
\toprule
 VUs &   Post &  Komentarz &  Logowanie &  Lekcja \\
\midrule
 100 & 1724.3 &     1755.0 &      470.8 &  1420.0 \\
 250 & 2688.8 &     3022.2 &      616.9 &  2165.1 \\
 600 & 3027.5 &     3406.8 &      641.0 &  2528.0 \\
1000 & 3717.2 &     4350.7 &      810.9 &  3234.7 \\
\bottomrule
\end{tabular}
\end{table}

\endgroup
p90 pozostaje przewidywalnie wyższy od median, ale utrzymuje komfortowe zakresy.

\begingroup
  \let\oldcaption\caption
  \renewcommand{\caption}[2][]{\oldcaption[#1]{#2. Źródło: opracowanie własne}}
  \begin{table}[H]
\centering
\caption{Minimum czasu odpowiedzi (ms) — wiersze: VUs, kolumny: endpoint}
\label{tab:min-vs-vus}
\begin{tabular}{rrrrr}
\toprule
 VUs &  Post &  Komentarz &  Logowanie &  Lekcja \\
\midrule
 100 &    24 &          4 &          1 &      16 \\
 250 &    57 &          4 &          1 &      48 \\
 600 &    61 &          3 &          1 &      77 \\
1000 &    82 &          4 &          1 &     113 \\
\bottomrule
\end{tabular}
\end{table}

\endgroup
Minimalne czasy potwierdzają niski czas przetwarzania, ścieżki są zoptymalizowane.

\begingroup
  \let\oldcaption\caption
  \renewcommand{\caption}[2][]{\oldcaption[#1]{#2. Źródło: opracowanie własne}}
  \begin{table}[H]
\centering
\caption{Maksimum czasu odpowiedzi (ms) — wiersze: VUs, kolumny: endpoint}
\label{tab:max-vs-vus}
\begin{tabular}{rrrrr}
\toprule
 VUs &  Post &  Komentarz &  Logowanie &  Lekcja \\
\midrule
 100 &  3261 &       3434 &       1288 &    2238 \\
 250 &  4277 &       5342 &       1498 &    3217 \\
 600 &  4700 &       5754 &       2129 &    3940 \\
1000 &  5814 &       7197 &       1726 &    4649 \\
\bottomrule
\end{tabular}
\end{table}

\endgroup
Maksymalne wartości są pojedynczymi przypadkami i nie wpływają na ogólne odczucie podczas używania aplikacji.


\FloatBarrier


% ===================== Podsumowanie =====================
\subsection{Podsumowanie testów wydajnościowych i jakościowych}

Przeprowadzony zestaw testów obejmował trzy komplementarne poziomy jakości:
\textbf{(1) testy jednostkowe} logiki domenowej (w tym serwisy, walidacja, operacje na repozytoriach Hazelcast),
\textbf{(2) testy integracyjne end-to-end} z użyciem wbudowanego serwera i klienta HTTP,
oraz \textbf{(3) testy wydajnościowe} w JMeter \cite{jmeter-docs}.
Takie ułożenie testów pozwoliło jednocześnie weryfikować poprawność funkcjonalną
i obserwować zachowanie systemu pod obciążeniem.

\paragraph{Poprawność funkcjonalności (jednostkowe \& integracyjne).}
Testy jednostkowe potwierdziły spójność logiki serwisów oraz prawidłowość filtracji w bazie danych.
Testy integracyjne end-to-end odwzorowały typowe działania podejmowane przez użytkowników (logowanie, dodawanie treści, interakcje użytkownika),
wskazując na poprawną współpracę warstw (kontrolery~$\leftrightarrow$~serwisy~$\leftrightarrow$~baza danych) oraz
na prawidłową obsługę sesji i CSRF. Operacje CRUD przebiegały zgodnie z oczekiwaniami, a odpowiedzi HTTP
zwracały oczekiwane kody i treści.

\paragraph{Wydajność i skalowalność (100--1000 VU).}
W testach obciążeniowych zróżnicowano liczbę równoczesnych użytkowników (100, 250, 600, 1000 VUs),
co pozwoliło zbadać zachowanie systemu pod rosnącym obciążeniem bez wprowadzania długotrwałego stresu.
Zebrane metryki (Mediana, Średnia, p90, Min, Max) wskazują na:
\begin{itemize}
  \item \textbf{Stabilny wzrost opóźnień} wraz ze wzrostem obciążenia, mediany rosną powoli,
        a p90 pozostaje w dobrym odstępie od mediany (Rys. \ref{fig:line-post-summary}).
  \item \textbf{Najkrótsze czasy dla Logowania}, co jest spójne z charakterem operacji (niewielka ilość przesłanych i otrzymanych danych).
        Operacje dodanie Posta i Komentarza są nieco większe, ale utrzymują niskie mediany oraz korzystne p90.
  \item \textbf{Lekcja} (dodanie lekcji) prezentuje trend porównywalny z dodaniem posta, stabilny bez skokowych zmian.
  \item \textbf{wartości Minimalne} wartości minimalnego czasu odpowiedzi potwierdzają, że ścieżka przetwarzania jest zoptymalizowana.
\item \textbf{wartości maksymalne} występują  sporadycznie, i nie wpływają na komfort użytkowania aplikacji
        (mediana i p90 mają niskie wartości).
\end{itemize}

\paragraph{Odczyt wyników na wykresach i w tabelach.}
\emph{Wykresy liniowe per operacja} (np. Rys. \ref{fig:line-post-summary}) pokazują,
że dla całego zakresu 100--1000 VU zachowany jest przewidywalny trend i komfortowe czasy odpowiedzi
dla wszystkich pięciu metryk. Wykresy słupkowe dla poszczególnych obciążeń
konsekwentnie potwierdzają te obserwacje. Tabele:
\textbf{Median} (Tab.~\ref{tab:mediana-vs-vus}),
\textbf{Średnich wartości} (Tab.~\ref{tab:srednia-vs-vus}),
\textbf{p90} (Tab.~\ref{tab:p90-vs-vus}),
\textbf{Minimalnych wartości} (Tab.~\ref{tab:min-vs-vus}) i
\textbf{maksymalnych wartości} (Tab.~\ref{tab:max-vs-vus})
Tabele przedstawiają zbiorcze zestawienie wyników, w którym wiersze odpowiadają liczbie wirtualnych użytkowników (VUs), a kolumny – poszczególnym operacjom. Potwierdzają trendy z wykresów.

\paragraph{Wnioski techniczne.}
Zastosowanie Hazelcast jako bazy danych sprzyja niskim opóźnieniom i stałości czasów odpowiedzi,
co widać szczególnie po medianie i p90. Architektura aplikacji wykazuje dobrą skalowalność wertykalną. W testowanym zakresie obciążenia, wzrost liczby wirtualnych użytkowników (VUs) nie ujawnił wąskich gardeł ani punktów saturacji. Profile wydajnościowe poszczególnych operacji pozostają spójne, co stanowi solidną podstawę do skalowania horyzontalnego (poprzez dodawanie kolejnych instancji aplikacji i węzłów Hazelcast) z zachowaniem niskich czasów odpowiedzi \cite{microservices}.

\paragraph{Konkluzja.}
W świetle przeprowadzonych testów można stwierdzić że,
aplikacja jest funkcjonalnie poprawna i spójna,
wydajna i przewidywalna,
gotowa do skalowania wraz z rosnącym ruchem.
Uzyskane charakterystyki wskazują na dojrzałość rozwiązania i jego przydatność do obsługi
współbieżnego ruchu w kontekście udostępniania materiałów edukacyjnych. % chapter6.tex zawiera treść rozdziału 6


\chapter{Podsumowanie projektu i wnioski}
W ramach zrealizowanego projektu stworzono funkcjonalną platformę webową służącą do udostępniania materiałów edukacyjnych, zarządzania kursami oraz interakcji między studentami a nauczycielami. Aplikacja została zaprojektowana i wykonana w oparciu o nowoczesny stos technologiczny (Spring Boot 3, Java 23, Thymeleaf, Bootstrap) z nietypowym rozwiązaniem w zakresie przechowywania danych użyciem bazy danych w pamięci Hazelcast zamiast tradycyjnej bazy relacyjnej. W toku implementacji osiągnięto założone cele funkcjonalne: użytkownicy mogą się rejestrować i logować, nauczyciele tworzyć kursy i materiały, studenci zapisywać na zajęcia i przesyłać zadania, a administratorzy nadzorować całość systemu i analizować statystyki. Rezultatem prac jest prototyp platformy e-learningowej o prostej, przejrzystej obsłudze, który może stanowić podstawę do dalszego rozwoju lub wdrożenia na potrzeby uczelni czy kursów online. W trakcie testów potwierdzono poprawność i stabilność działania aplikacji, co oznacza, że przyjęte decyzje architektoniczne i technologiczne sprawdziły się w praktyce. Na zakończenie, warto dokonać analizy kluczowej decyzji projektowej dotyczącej bazy danych: zastosowania bazy danych w pamięci zamiast klasycznej bazy relacyjnej. Poniżej zebrano główne zalety i wady takiego rozwiązania zaobserwowane w kontekście tego projektu: \textbf{Zalety wykorzystania bazy in-memory (Hazelcast) w porównaniu z relacyjną bazą danych:}
\begin{itemize}
\item \textbf{Wysoka wydajność i niskie opóźnienia dostępu do danych} – wszystkie dane przechowywane są w pamięci RAM, dzięki czemu odczyty i zapisy odbywają się bez opóźnień związanych z operacjami dyskowymi. W testach aplikacja z Hazelcastem cechowała się bardzo krótkimi czasami odpowiedzi, co pozytywnie wpływa na doświadczenie użytkowników.
\item \textbf{Prostsza skalowalność pozioma} – Hazelcast z natury wspiera klasteryzację i replikację danych między węzłami. Dodanie kolejnego serwera aplikacji automatycznie zwiększa dostępną pamięć i moc obliczeniową klastra Hazelcast, co umożliwia obsługę większej liczby użytkowników bez skomplikowanej konfiguracji klastrów baz danych.
\item \textbf{Łatwość wdrożenia i brak zależności od zewnętrznej infrastruktury DBMS} – aplikacja nie wymaga instalacji i administrowania osobnym serwerem bazy danych (np. MySQL/PostgreSQL). Hazelcast działa wbudowany w aplikację, co upraszcza proces uruchomienia platformy na nowym środowisku  wystarczy uruchomić aplikację Java, a magazyn danych inicjuje się samoczynnie.
\item \textbf{Wsparcie dla dodatkowych funkcjonalności} – Hazelcast to nie tylko magazyn klucz-wartość. Udostępnia wiele zaawansowanych struktur i usług (kolejki, mechanizmy locków, strumieniowe przetwarzanie danych itp.). W projekcie wykorzystano np. \texttt{FlakeIdGenerator} do generacji identyfikatorów czy \texttt{IMap} z indeksami do szybkiego wyszukiwania. Te mechanizmy w pewnych scenariuszach mogą upraszczać implementację (omijając konieczność pisania złożonych zapytań SQL).
\end{itemize} \textbf{Wady i ograniczenia podejścia in-memory względem tradycyjnej bazy relacyjnej:}
\begin{itemize}
\item \textbf{Brak trwałości danych (domyślnie)} – podstawowym minusem jest ulotność danych. O ile bazy relacyjne zapewniają domyślnie trwały zapis na dysku , o tyle Hazelcast przechowuje dane w pamięci awaria lub restart skutkuje utratą danych, chyba że zastosuje się dodatkowe mechanizmy (np. zapis  na dysku). W kontekście aplikacji edukacyjnej utrata danych mogłaby być krytyczna (np. zniknięcie informacji o kontach czy zadaniach), więc w środowisku produkcyjnym należałoby tę kwestię rozwiązać przed wdrożeniem.
\item \textbf{Zużycie pamięci operacyjnej} – przechowywanie kompletnej bazy danych w RAM może być kosztowne pamięciowo, zwłaszcza przy dużej skali danych (tysiące użytkowników, setki tysięcy rekordów). Pamięć jest droższym zasobem niż przestrzeń dyskowa, zatem aplikacja oparta wyłącznie na Hazelcast może wymagać serwerów z dużą ilością RAM, aby pomieścić wszystkie dane. W przeciwnym razie konieczne byłoby ograniczanie zakresu danych (np. archiwizacja starszych informacji poza Hazelcastem).
\item \textbf{Ograniczone możliwości zapytań} – relacyjne bazy danych i język SQL oferują bardzo bogate możliwości zapytań (łączenie tabel, agregacje, złożone filtrowanie, podzapytania itp.). W Hazelcast również istnieje możliwość wykonywania zapytań (nawet zbliżonych do SQL przy użyciu Hazelcast SQL), jednak nie jest to tak elastyczne i wydajne przy bardzo złożonych zapytaniach jak w wyspecjalizowanych silnikach SQL. W moim projekcie model danych był stosunkowo prosty (brak skomplikowanych relacji), więc ta wada nie była odczuwalna, ale w bardziej złożonym domenowo systemie mogłaby stanowić problem.
\item \textbf{mniej popularne rozwiązanie} – rozwiązania oparte o relacyjne bazy danych są dobrze znane programistom i administratorom, istnieje wiele materiałów, narzędzi i dobrych praktyk. Hazelcast, będący specyficzną bazą danych w pamięci, może wymagać od zespołu pewnej krzywej nauki oraz ostrożności w użyciu (np. konfiguracja klastrów).
\end{itemize} Podsumowując, zastosowanie Hazelcast w projekcie przyniosło wymierne korzyści w zakresie wydajności i uproszczenia warstwy dostępu do danych, co było istotne z punktu widzenia prototypowania i demonstracji działania platformy. Z drugiej strony, zdano sobie sprawę z ograniczeń takiego podejścia w szczególności kwestii trwałości danych co należałoby adresować przed użyciem systemu w środowisku produkcyjnym. Możliwym kierunkiem rozwoju byłoby wprowadzenie hybrydowej architektury, gdzie Hazelcast działałby jako warstwa cache i mechanizm szybkiego dostępu (np. do sesji i często czytanych danych), natomiast ostateczna trwała kopia danych byłaby przechowywana w relacyjnej bazie danych w sposób asynchroniczny. Projekt zaliczyć można do udanych – zrealizował zakładane funkcjonalności, a przy tym stanowi ciekawy przykład wykorzystania nowatorskich rozwiązań w aplikacjach webowych. Wnioski płynące z prac nad nim mogą być cenne przy podejmowaniu decyzji architektonicznych w przyszłych systemach: wybór bazy danych powinien być podyktowany konkretnymi wymaganiami i ograniczeniami projektu (wydajność vs. trwałość, prostota vs. możliwości zapytań).

%----------------------------------------------------------------------------------------
%	ROZDZIAŁy kolejne należy dodać analogicznie do 1 i 2 
% utworzyć pliki i je załączyć (include)
%----------------------------------------------------------------------------------------

%----------------------------------------------------------------------------------------
%	ZAKOŃCZENIE PRACY DYPLOMOWEJ (JK - design and implementation)
%----------------------------------------------------------------------------------------
\addcontentsline{toc}{chapter}{Spis Literatury}



%----------------------------------------------------------------------------------------
%	SPIS LITERATURY (JK - design and implementation)
%----------------------------------------------------------------------------------------
% \pagestyle{empty} 
% \chapter*{Spis literatury}
% \addcontentsline{toc}{chapter}{\textcolor{blueZUT}%
% Poniżej zdefiniowane są filtry do dzielenia spisu literatury na kategorie
% % (removed broken bib filters)
% % (removed broken bib filters)

% UWAGA! -aby zmienić zawartość spisu literatury należy wyedytować plik


%------------------------------------------------
% Spis literartury podzielony jest na 3 kategorie
% 1
% \section*{Książki}
% \addcontentsline{toc}{section}{Książki}
%% (removed include of bibliography.tex; using biblatex\printbibliography)

%------------------------------------------------
% 2
% \section*{Artykuły}
% \addcontentsline{toc}{section}{Artykuły}
% \printbibliography[heading=bibempty,filter=articles]
%------------------------------------------------
% 3
% \section*{Źródła internetowe i inne}
% \addcontentsline{toc}{section}{Źródła internetowe i inne}
\printbibliography


%----------------------------------------------------------------------------------------
%	APPENDIX (JK - design and implementation)
%----------------------------------------------------------------------------------------
\pagestyle{fancy} 
\begin{appendix}
\appendix
%%%%%%%%%%%%%%%%%%%%%%%%%%%%%%%%%%%%%%%%%
% Szablon pracy dyplomowej
% Wydział Informatyki 
% Zachodniopomorski Uniwersytet Technologiczny w Szczecinie
% autor Joanna Kołodziejczyk (jkolodziejczyk@zut.edu.pl)
% Bardzo wczesnym pierwowzorem szablonu był
% The Legrand Orange Book
% Version 5.0 (29/05/2025)2023)
%
% Modifications to LOB assigned by %JK
%%%%%%%%%%%%%%%%%%%%%%%%%%%%%%%%%%%%%%%%%
\thispagestyle{empty}
\listoffigures

\chapter{Dodatek}
\label{chapter:dodatek_A}

\section*{Cel}
Niniejszy aneks zawiera zwięzłe, implementacyjne podsumowanie aplikacji webowej zrealizowanej w ramach przedstawionej pracy. System umożliwia udostępnianie, wyszukiwanie i ocenę materiałów edukacyjnych oraz pokazuje, w jaki sposób baza danych w pamięci Hazelcast może poprawić responsywność i skalowalność aplikacji opartej na Spring Boot.

\section*{Przegląd systemu}
Aplikacja stosuje warstwowy, zorientowany na usługi styl projektowy. Warstwa back-end jest zaimplementowana w Javie (Spring Boot) i udostępnia REST-owe endpointy dla interfejsu użytkownika. Hazelcast wykorzystywany jest jako baza danych.

\begin{table}[h]
  \centering
  \caption{Główne komponenty i odpowiedzialności}
  \label{tab:components}
  \begin{tabular}{@{}p{3.3cm}p{9.2cm}@{}}
  \toprule
  Komponent & Odpowiedzialność \\
  \midrule
  Kontrolery & Endpointy HTTPS dla autentykacji, materiałów, wyszukiwania oraz funkcji użytkownika/profilu. \\
  Auth \& Security & Weryfikacja , obsługa sesji/JWT, kontrola dostępu oparta na rolach (user/admin). \\
  Wyszukiwarka & Wyszukiwanie słów kluczowych po zcache'owanych metadanych i tagach. \\
  Rating/Comment Service & feedback od użytkowników. \\
  Admin & przegląd nowych zgłoszeń, akceptowanie i odrzucanie zgłoszeń, usuwanie/edytowanie treści. \\
  \bottomrule
  \end{tabular}
\end{table}

\section*{Wybrane przypadki użycia}
\begin{itemize}
  \item \textbf{Rejestracja użytkownika:} Gość podaje imię, e-mail i hasło; system waliduje dane, sprawdza unikalność e-maila.
  \item \textbf{Logowanie i sesja:} Istniejący użytkownik uwierzytelnia się; wydawana jest sesja.
  \item \textbf{Przesyłanie materiału:} Zalogowany użytkownik wgrywa plik z tytułem i opisem.
  \item \textbf{Przeglądanie artykułów:} Każdy użytkownik może przeglądać stronnicowane listy artykułów.
  \item \textbf{Wyszukiwanie materiałów:} Użytkownik wyszukuje po słowach kluczowych; usługa odpytuje indeks w pamięci, zwracając pasujące zasoby.
  \item \textbf{Podgląd/Pobranie:} Użytkownik otwiera stronę z dostępnym materiałem do pobrania klika przycisk pobierz, udostępniony plik zostaje pobrany
  \item \textbf{Ocena i komentarz:} Pod artykułami użytkownicy wystawiają oceny (1--5) oraz mogą publikować komentarze.
  \item \textbf{Zarządzanie profilem:} Użytkownik aktualizuje dane na stronie do zarządzania profilem
  \item \textbf{Moderacja administracyjna:} Administrator zatwierdza/odrzuca oczekujące zgłoszenia do klas oraz zarządza użytkownikami i artykulami.
\end{itemize}

\section*{przykład REST API}
\noindent Poniższa tabela szkicuje przykładowe REST endpointy. O ile nie wskazano inaczej, wymagane jest uwierzytelnienie.

\begin{table}[h]
  \centering
  \caption{Przykładowe punkty końcowe}
  \label{tab:endpoints}
  \begin{tabular}{@{}p{2.1cm}p{6.4cm}p{3.9cm}@{}}
  \toprule
  Metoda & Ścieżka & Cel / Autoryzacja \\
  \midrule
  POST & \texttt{/api/auth/register} & Utworzenie konta, dostępne dla każdego. \\
  POST & \texttt{/api/auth/login} & Uzyskanie sesji/JWT, dostępne dla każdego. \\
  GET & \texttt{/api/posts} & przegląd postów, dostępne dla każdego. \\
  GET & \texttt{/api/posts/\{id\}} & szczegóły posta; dostępne dla każdego. \\
  POST & \texttt{/api/posts} & Przesłanie nowego materiału, dostępne dla nauczyciela i admina. \\
  GET & \texttt{/api/search?q=...} & Wyszukiwanie postów; dostępne dla każdego. \\
  POST & \texttt{/api/posts/\{id\}/ratings} & Dodanie oceny, dostępne dla zalogowanych. \\
  POST & \texttt{/api/posts/\{id\}/comments} & Dodanie komentarza; dostępne dla zalogowanych. \\
  \bottomrule
  \end{tabular}
\end{table}

\section*{Konfiguracja — wycinek (Hazelcast)}
\noindent Poniższe zestawienie pokazuje minimalny kod konfiguracji Spring Boot uruchamiający instancję Hazelcast:

\begin{lstlisting}[language=yaml,caption={Minimalny fragment konfiguracji Hazelcast},label={lst:hz-config}]
# application.yml
spring:
  cache:
    type: hazelcast
  session:
    store-type: none
# przyklad konfiguracji
# @Bean
# public HazelcastInstance hazelcastInstance() {
#   Config cfg = new Config();
#   cfg.setClusterName("edu-share-cluster");
#   cfg.addMapConfig(new MapConfig("materials")
#       .setTimeToLiveSeconds(0)     
#       .setBackupCount(1));          
#   cfg.addMapConfig(new MapConfig("sessions")
#       .setTimeToLiveSeconds(1800)); # 30 min TTL sesji, jeśli używane
#   return Hazelcast.newHazelcastInstance(cfg);
# }
\end{lstlisting}

\section*{Atrybuty jakości i uzasadnienie}
\textbf{Wydajność i skalowalność:} Ścieżki z przewagą odczytów, korzystają z dostępu w pamięci, co skraca czasy odpowiedzi i odciąża bazę. Skalowanie horyzontalne wspiera dystrybucja danych i klaster Hazelcast

\noindent\textbf{Bezpieczeństwo:} Stosowane są standardowe wzorce Spring Security (haszowanie haseł, autoryzacja oparta na rolach). Operacje wrażliwe (moderacja) wymagają uwierzytelnienia i odpowiednich ról.

\noindent\textbf{Utrzymywalność:} Wyraźny podział odpowiedzialności (kontrolery, serwisy, repozytoria) oraz dobrze zdefiniowane REST-owe endpointy ułatwiają testowanie i przyszłą rozbudowę (np. nowe typy treści).

\section*{Kierunki rozwoju}
Planowane usprawnienia obejmują: lepsza wyszukiwarka, podgląd treści, zaawansowane strategie cache'owania optymalizujące zużycie pamięci przy dużym obciążeniu.

 % conclusions.tex zawiera treść zakończenia/podsumowania pracy dyplomowej
\end{appendix}
%----------------------------------------------------------------------------------------
% Koniec dokumentu
\end{document}
