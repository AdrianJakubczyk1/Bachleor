%%%%%%%%%%%%%%%%%%%%%%%%%%%%%%%%%%%%%%%%%
% Specjalna strona pracy ze streszczeniem i abstractem w j. angielskim
% Szablon pracy dyplomowej
% Wydział Informatyki 
% Zachodniopomorski Uniwersytet Technologiczny w Szczecinie
% autor Joanna Kołodziejczyk (jkolodziejczyk@zut.edu.pl)
% Bardzo wczesnym pierwowzorem szablonu był
% The Legrand Orange Book
% Version 5.0 (29/05/2025)
%
% Modifications to LOB assigned by %JK
%%%%%%%%%%%%%%%%%%%%%%%%%%%%%%%%%%%%%%%%%


\begin{center}
\noindent {{\color{blueZUT}\Large\sffamily  {Streszczenie}}}\\[1cm] 



Celem pracy było przeprowadzenie analizy porównawczej wiodących systemów baz danych w pamięci oraz, na jej podstawie, opracowanie aplikacji w języku Java udostępniającej materiały dydaktyczne.
W wyniku analizy systemów Memcached, Redis, Apache Ignite i Hazelcast, do implementacji wybrano ten ostatni, w którym wszystkie dane (lekcje, zadania, użytkownicy) przechowywane są w klastrowych mapach pamięciowych, co zapewnia niskie opóźnienia i łatwą skalowalność.
W części teoretycznej przedstawiono kryteria oceny i przeprowadzono szczegółowe porównanie architektur, modeli danych i wydajności analizowanych systemów.
Część praktyczna obejmuje projekt i implementację aplikacji w oparciu o Spring Boot 3.4.1, realizację operacji CRUD na strukturach `IMap`, a także budowę paneli administratora i nauczyciela oraz mechanizmów interakcji użytkowników.
Przeprowadzone testy wydajnościowe i integracyjne potwierdziły słuszność wyboru architektonicznego, wykazując wysoką wydajność i poprawność działania rozwiązania.
Zaproponowana aplikacja stanowi skalowalną podstawę do dalszego rozwoju.


\end{center}

