%%%%%%%%%%%%%%%%%%%%%%%%%%%%%%%%%%%%%%%%%
% Specjalna strona pracy ze streszczeniem i abstractem w j. angielskim
% Szablon pracy dyplomowej
% Wydział Informatyki 
% Zachodniopomorski Uniwersytet Technologiczny w Szczecinie
% autor Joanna Kołodziejczyk (jkolodziejczyk@zut.edu.pl)
% Bardzo wczesnym pierwowzorem szablonu był
% The Legrand Orange Book
% Version 5.0 (29/05/2025)
%
% Modifications to LOB assigned by %JK
%%%%%%%%%%%%%%%%%%%%%%%%%%%%%%%%%%%%%%%%%


\begin{center}
\noindent {{\color{blueZUT}\Large\sffamily  {Streszczenie}}}\\[1cm] 
\end{center}
Celem pracy było zaprojektowanie oraz zaimplementowanie aplikacji webowej umożliwiającej efektywne współdzielenie materiałów edukacyjnych. Przedstawiona aplikacja została zrealizowana z wykorzystaniem technologii Spring oraz bazy danych działającej w pamięci operacyjnej (in-memory) – H2 oraz relacyjna baza danych PostgreSQL. W części teoretycznej omówiono wykorzystane technologie, w tym framework Spring, bazę danych H2, baze danych PostgreSQL oraz wzorce projektowe wykorzystywane podczas budowy aplikacji internetowych.

W ramach części praktycznej zaprojektowano strukturę danych, interfejs użytkownika oraz zaimplementowano funkcjonalności takie jak publikacja materiałów, możliwość oceniania oraz komentowania treści przez użytkowników, zarządzanie profilami ,panel administracyjny, panel nauczyciela oraz możliwość polubienia i ocenienia materiałów. Do przechowywania tymczasowych danych statystycznych wykorzystano technologię schedulera, który okresowo aktualizuje statystyki aplikacji. Dodatkowo aplikacja zawiera mechanizmy bezpieczeństwa oraz uwierzytelniania użytkowników z wykorzystaniem Spring Security.

W wyniku przeprowadzonych testów aplikacja spełniła założone cele projektowe, zapewniając intuicyjny dostęp do materiałów edukacyjnych oraz łatwe zarządzanie treścią dla administratorów. Przedstawione rozwiązanie może stanowić podstawę do dalszego rozwoju lub integracji z innymi systemami edukacyjnymi.

\vfill

\begin{center}
\noindent {{\color{blueZUT}\Large\sffamily {Abstract}}}\\[1cm] 
\end{center}
The abstract's purpose, which should not exceed 150 words, is to provide sufficient information to allow potential readers to decide on the thesis's relevance—a maximum of half the page.

\vspace{10pt}
\noindent{\bf keywords:} e.g.: computer science, control, computer graphics