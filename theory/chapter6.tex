\chapter{Testowanie}

Rozdział ten opisuje proces weryfikacji jakości stworzonej aplikacji, obejmujący testy jednostkowe, integracyjne oraz testy manualne.

\section{Testy jednostkowe}

W testach jednostkowych wykorzystano:

\begin{itemize}
    \item \textbf{JUnit 5} – podstawowy framework testowy.
    \item \textbf{Mockito} – narzędzie do izolowania zależności między komponentami.
\end{itemize}

Przykładowy test jednostkowy przedstawia sprawdzanie poprawności aktualizacji statystyk:

\begin{lstlisting}[language=Java]
@Test
void shouldUpdateStatisticsCorrectly() {
    when(postRepository.count()).thenReturn(10L);
    statsService.updateStatistics();
    verify(statsDao, times(1)).saveStat(any(AppStatistic.class));
}
\end{lstlisting}

\section{Testy integracyjne}

Testy integracyjne realizowano z wykorzystaniem bazy danych H2, co zapewniło izolowane i szybkie środowisko testowe. Wykorzystano narzędzia:

\begin{itemize}
    \item \textbf{Spring Boot Test}
    \item \textbf{TestRestTemplate}
\end{itemize}

Przykład testu integracyjnego:

\begin{lstlisting}[language=Java]
@Test
void homePageLoadsCorrectly() {
    ResponseEntity<String> response = restTemplate.getForEntity("/", String.class);
    assertEquals(HttpStatus.OK, response.getStatusCode());
}
\end{lstlisting}

\section{Testowanie manualne}
