%%%%%%%%%%%%%%%%%%%%%%%%%%%%%%%%%%%%%%%%%
% Szablon pracy dyplomowej
% Wydział Informatyki 
% Zachodniopomorski Uniwersytet Technologiczny w Szczecinie
% autor Joanna Kołodziejczyk (jkolodziejczyk@zut.edu.pl)
% Bardzo wczesnym pierwowzorem szablonu był
% The Legrand Orange Book
% Version 5.0 (29/05/2025)
%
% Modifications to LOB assigned by %JK
%%%%%%%%%%%%%%%%%%%%%%%%%%%%%%%%%%%%%%%%%

%----------------------------------------------------------------------------------------
%	CHAPTER 1
% 	author: Joanna Kolodziejczyk (jkolodziejczyk@zut.edu.pl)
%----------------------------------------------------------------------------------------

\chapter{Technologie zastosowane w projekcie}
\label{rozdzial1}

Celem niniejszego rozdziału jest przedstawienie technologii oraz narzędzi wykorzystanych w procesie implementacji aplikacji do udostępniania materiałów edukacyjnych. W rozdziale opisano przede wszystkim technologię Spring Framework oraz użyte systemy baz danych: PostgreSQL jako bazę danych produkcyjną oraz H2, która pełni rolę pomocniczą, przechowującą dane statystyczne, testowe oraz sesyjne.

\section{Spring Framework}

Spring Framework jest obecnie jednym z najpopularniejszych narzędzi do tworzenia aplikacji Java klasy enterprise. Charakteryzuje się modularnością, łatwą integracją z innymi technologiami oraz obsługą odwrócenia kontroli (ang. \textit{Inversion of Control - IoC}) oraz wstrzykiwania zależności (ang. \textit{Dependency Injection - DI}). Szczególną uwagę zwraca Spring Boot, będący rozszerzeniem podstawowego frameworka, który umożliwia szybką konfigurację oraz uruchomienie aplikacji, dostarczając gotowe rozwiązania wielu typowych problemów programistycznych.

W projekcie wykorzystano między innymi:
\begin{itemize}
    \item \textbf{Spring MVC} - do implementacji warstwy widoku i obsługi żądań HTTP.
    \item \textbf{Spring Security} - do zarządzania uwierzytelnianiem użytkowników oraz autoryzacją dostępu do zasobów aplikacji.
    \item \textbf{Spring Data JDBC} - do zarządzania dostępem do baz danych za pomocą wygodnych repozytoriów oraz operacji CRUD.
\end{itemize}

\section{Baza danych PostgreSQL}

Do trwałego przechowywania danych produkcyjnych aplikacji użyta została baza danych PostgreSQL, która charakteryzuje się wysoką wydajnością, zgodnością ze standardami SQL oraz wsparciem dla zaawansowanych operacji i funkcjonalności. W ramach aplikacji PostgreSQL przechowuje takie informacje, jak dane użytkowników, publikowane materiały, komentarze, oceny i inne dane związane z funkcjonalnościami aplikacji.

\section{Baza danych H2}

W projekcie wykorzystano również wbudowaną bazę danych H2, działającą w pamięci operacyjnej. Jej zastosowanie obejmuje:
\begin{itemize}
    \item \textbf{Testowanie aplikacji} - baza danych H2 pozwala na szybkie i efektywne przeprowadzanie testów integracyjnych oraz jednostkowych.
    \item \textbf{Przechowywanie danych sesyjnych} - H2 umożliwia szybkie zarządzanie danymi sesyjnymi aplikacji, dzięki czemu aplikacja może działać szybciej i sprawniej obsługiwać wiele jednoczesnych sesji.
    \item \textbf{Statystyki aplikacji} - baza danych H2 przechowuje także dane dotyczące statystyk aplikacji, które są wykorzystywane do analizowania aktywności użytkowników i działania systemu.
\end{itemize}

Zastosowanie dwóch rodzajów baz danych (PostgreSQL i H2) pozwala na optymalne wykorzystanie ich cech, zwiększenie wydajności aplikacji oraz uproszczenie procesu rozwoju, testowania i wdrażania.

\section{Inne technologie}

Oprócz wymienionych wyżej technologii w projekcie zastosowano dodatkowo:
\begin{itemize}
    \item \textbf{Thymeleaf} - do renderowanie stron HTML w aplikacji webowej.
    \item \textbf{Bootstrap 5} - do stylizacji interfejsu użytkownika oraz zapewnienia responsywności aplikacji.
    \item \textbf{JUnit} - do pisania testów jednostkowych oraz integracyjnych.
    \item \textbf{Maven} - do zarządzania zależnościami oraz budowania aplikacji.
    \item \textbf{Javscript} - do zarządzania wyglądem strony oraz do technologii AJAX
    \item \textbf{CSS} - do zarządzania wyglądem strony
\end{itemize}

W kolejnych rozdziałach przedstawione zostaną szczegóły implementacyjne dotyczące wyżej wymienionych technologii oraz sposób ich wykorzystania w praktyce.





