%%%%%%%%%%%%%%%%%%%%%%%%%%%%%%%%%%%%%%%%%
% Szablon pracy dyplomowej
% Wydział Informatyki 
% Zachodniopomorski Uniwersytet Technologiczny w Szczecinie
% autor Joanna Kołodziejczyk (jkolodziejczyk@zut.edu.pl)
% Bardzo wczesnym pierwowzorem szablonu był
% The Legrand Orange Book
% Version 5.0 (29/05/2025)
%
% Modifications to LOB assigned by %JK
%%%%%%%%%%%%%%%%%%%%%%%%%%%%%%%%%%%%%%%%%

%----------------------------------------------------------------------------------------
%	CHAPTER 1
% 	author: Joanna Kolodziejczyk (jkolodziejczyk@zut.edu.pl)
%----------------------------------------------------------------------------------------

\chapter{Technologie zastosowane w projekcie}
\label{rozdzial1}

Celem niniejszego rozdziału jest przedstawienie technologii oraz narzędzi wykorzystanych w procesie implementacji aplikacji do udostępniania materiałów edukacyjnych. W rozdziale opisano przede wszystkim technologię Spring Framework oraz zastosowaną bazę danych Hazelcast.

\section{Język Java i Platforma JVM}

Java to język ogólnego przeznaczenia o statycznym typowaniu, którego programy uruchamia \emph{Java Virtual Machine} (JVM). Kompilacja do bajtkodu i wykonanie na JVM sprawiają, że ten sam artefakt może działać na różnych systemach operacyjnych bez rekompilacji.  \cite{java-docs}.

\subsection*{JDK, standardowa biblioteka i model wykonania}
Środowisko deweloperskie (JDK) dostarcza narzędzia takie jak kompilator \texttt{javac} oraz rozbudowaną bibliotekę standardową. Znajdują się tam m.in.\ kolekcje, strumienie wejścia/wyjścia, API sieciowe. W praktyce oznacza to, że wiele typowych zadań aplikacyjnych można zrealizować bez dodatkowych zależności. \cite{java-docs,java-head-first}.

\subsection*{Zarządzanie pamięcią i bezpieczeństwo}
JVM automatycznie odzyskuje nieużywaną pamięć poprzez Garbage Collector. Dzięki Garbage Collector nie trzeba się martwić zarządzaniem pamięcią \cite{java-docs}.


\subsection*{Dobre praktyki projektowe}
W codziennej pracy istotne są zasady dotyczące m.in.\ niezmienności, poprawnego nadpisywania \texttt{equals}/\texttt{hashCode} i bezpiecznego publikowania obiektów. Takie wytyczne pomagają unikać subtelnych błędów oraz poprawiają czytelność i utrzymywalność kodu w większych bazach kodu \cite{effective-java-3e}.

\section{Spring Framework}

Spring stanowi szeroko wykorzystywany zestaw komponentów do tworzenia aplikacji w języku Java. Framework dostarcza mechanizmy wstrzykiwania zależności, wsparcie dla warstwowej architektury oraz integracje z powszechnie używanymi bibliotekami. W projekcie zastosowano Spring Boot, który upraszcza konfigurację i uruchamianie usług dzięki zestawowi starterów czyli gotowym modułom zależności, które automatycznie importują niezbędne biblioteki, frameworki i konfiguracje,co istotnie przyspiesza implementację i ułatwia utrzymanie rozwiązania \cite{spring-boot}.

W projekcie wykorzystano między innymi:

– \textbf{Spring MVC} –- do implementacji warstwy widoku i obsługi żądań HTTP.

– \textbf{Spring Security} -– do zarządzania uwierzytelnianiem użytkowników oraz autoryzacją dostępu do zasobów aplikacji.

– \textbf{Spring Data} -– do zarządzania dostępem do baz danych za pomocą wygodnych repozytoriów oraz operacji CRUD.

\section{Baza danych w pamięci Hazelcast}

W projekcie wykorzystano bazę danych w pamięci  – Hazelcast. Baza ta charakteryzuje się wyjątkową wydajnością, niskimi opóźnieniami oraz dużą skalowalnością dzięki możliwości pracy w klastrze. Najczęściej wykorzystywaną strukturą jest rozproszona mapa klucz–wartość (\texttt{IMap}), Hazelcast służy jako szybka warstwa przechowywania i wymiany danych pomiędzy komponentami systemu, przechowująca wszystkie dane aplikacji, takie jak informacje o użytkownikach, lekcjach, zadaniach, komentarzach, ocenach oraz dane sesyjne i statystyczne \cite{hazelcast-docs}. Wybór Hazelcast jako kluczowego komponentu warstwy danych został poprzedzony szczegółową analizą porównawczą alternatywnych rozwiązań, która została przedstawiona w kolejnym rozdziale.

Kluczowe zalety wykorzystania Hazelcast w projekcie:

– Wysoka wydajność -– wszystkie operacje na danych odbywają się w pamięci RAM, co zapewnia szybki dostęp do informacji.

– Skalowalność -– możliwość łatwego dodawania nowych węzłów do klastra w celu zwiększenia wydajności i pojemności systemu.

– Trwałość i odporność na awarie -– dzięki mechanizmowi kopi zapasowych (backup-count), dane są zabezpieczone przed utratą w razie awarii jednego z węzłów klastra.

– Bogate możliwości indeksowania oraz szybkiego wyszukiwania danych.

– Przechowywanie danych sesyjnych i tymczasowych bez konieczności stosowania dodatkowych rozwiązań.

Zastosowanie Hazelcast pozwoliło na znaczące uproszczenie architektury aplikacji oraz zwiększenie jej ogólnej wydajności i niezawodności.

\section{Inne technologie}

Oprócz wymienionych wyżej technologii w projekcie zastosowano dodatkowo kilka narzędzi i bibliotek, które wspomagają budowanie nowoczesnych aplikacji webowych oraz poprawiają jakość i komfort użytkowania.

- \textbf{Thymeleaf} –  silnik szablonów, który integruje się z technologią Spring. Umożliwia renderowanie stron HTML, automatycznie dostarczając dane z modelu aplikacji bezpośrednio do widoków. Ułatwia wiązanie danych z warstwą prezentacji i dobrze współpracuje ze Spring Boot \cite{thymeleaf-docs}.

– \textbf{Bootstrap 5} – popularny framework CSS, który zapewnia podstawowe style oraz komponenty, pozwalając na szybkie tworzenie responsywnych i estetycznych interfejsów użytkownika. Zapewnia między innymi spójne układy i typowe elementy nawigacyjne \cite{bootstrap-docs}.

– \textbf{JUnit} – biblioteka testowa przeznaczona do pisania i wykonywania testów, w projekcie wykorzystywano testy JUnit do sprawdzania poprawności działania poszczególnych komponentów systemu \cite{junit-docs}.


– \textbf{Maven} – narzędzie do zarządzania cyklem życia projektu Java (kompilacja, testowanie, pakowanie) oraz zależnościami, oparte na pliku \texttt{pom.xml} \cite{maven-docs}.

– \textbf{JavaScript} – język programowania stosowany po stronie przeglądarki, w projekcie JavaScript wykorzystywany był przede wszystkim do obsługi interakcji użytkownika, walidacji danych po stronie klienta oraz obsługi asynchronicznych żądań za pomocą technologii AJAX \cite{javascript-docs}.

– \textbf{CSS} – kaskadowe arkusze stylów odpowiedzialne za wygląd i układ elementów interfejsu, stosowane razem z Bootstrap w celu uzyskania spójnej warstwy wizualnej \cite{css-docs}.

W kolejnych rozdziałach przedstawione zostaną szczegóły implementacyjne dotyczące wyżej wymienionych technologii oraz sposób ich wykorzystania w praktyce.






