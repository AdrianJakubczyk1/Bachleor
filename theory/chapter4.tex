\chapter{Testowanie i ocena aplikacji}

W niniejszym rozdziale przedstawiono proces testowania aplikacji oraz ocenę poprawności jej działania. Omówione zostały podejścia wykorzystane w testach jednostkowych oraz integracyjnych, a także ocena wydajności aplikacji.

\section{Wprowadzenie do testowania aplikacji}

Testowanie stanowi kluczowy element procesu tworzenia oprogramowania, umożliwiający wykrycie błędów oraz weryfikację poprawności działania zaimplementowanych funkcjonalności. W ramach tej aplikacji wykorzystano dwa główne rodzaje testów:

\begin{itemize}
    \item Testy jednostkowe – testujące pojedyncze klasy oraz metody.
    \item Testy integracyjne – sprawdzające poprawność współdziałania wielu komponentów systemu.
\end{itemize}

\section{Testy jednostkowe}

Testy jednostkowe zostały napisane z użyciem frameworka JUnit 5, który jest szeroko stosowany w aplikacjach Java ze względu na łatwość integracji ze środowiskiem Spring Boot. 

Przykładowe testy jednostkowe objęły klasy:

\begin{itemize}
    \item Serwisy (np. \texttt{StatsService}), które weryfikują poprawność operacji na danych.
    \item Klasy DAO (np. \texttt{StatsDaoImpl}), które sprawdzają prawidłowe wykonanie zapytań SQL do bazy danych H2.
    \item Kontrolery, gdzie przy pomocy MockMVC testowano poprawność zwracanych widoków oraz danych.
\end{itemize}

\subsection{Przykład testu jednostkowego}

Poniżej przedstawiono przykład testu jednostkowego dla klasy \texttt{StatsService}:

\begin{lstlisting}[language=Java]
@Test
void shouldUpdateStatisticsCorrectly() {
    when(postRepository.count()).thenReturn(10L);
    when(userRepository.countByRole("TEACHER")).thenReturn(2L);

    statsService.updateStatistics();

    verify(statsDao, times(1)).saveStat(any(AppStatistic.class));
}
\end{lstlisting}

Wykorzystanie mechanizmów \texttt{Mockito} pozwoliło na izolację testowanych metod od zewnętrznych zależności.

\section{Testy integracyjne}

Testy integracyjne przeprowadzono z wykorzystaniem frameworka Spring Boot oraz wbudowanej bazy danych H2. Głównym celem było sprawdzenie poprawności współpracy różnych warstw aplikacji.

Do realizacji tych testów wykorzystano:

\begin{itemize}
    \item \texttt{TestRestTemplate} – do symulacji pełnych zapytań HTTP i sprawdzania zwracanych odpowiedzi.
    \item Spring Test – zapewniający kompleksową konfigurację środowiska testowego.
\end{itemize}

\subsection{Przykład testu integracyjnego}

Poniżej przedstawiono przykład testu integracyjnego sprawdzającego poprawność działania strony głównej aplikacji:

\begin{lstlisting}[language=Java]
@SpringBootTest(webEnvironment = SpringBootTest.WebEnvironment.RANDOM_PORT)
class ApplicationIntegrationTests {

    @Autowired
    private TestRestTemplate restTemplate;

    @Test
    void shouldLoadHomePageSuccessfully() {
        ResponseEntity<String> response = restTemplate.getForEntity("/", String.class);

        assertEquals(HttpStatus.OK, response.getStatusCode());
        assertTrue(response.getBody().contains("Educational Materials Sharing Platform"));
    }
}
\end{lstlisting}

Testy integracyjne potwierdziły poprawne współdziałanie komponentów aplikacji.

\section{Baza danych testowa (H2)}

W trakcie realizacji testów wykorzystano bazę danych H2 działającą w pamięci, która umożliwiła szybkie i bezpieczne wykonywanie operacji CRUD, bez ingerencji w dane produkcyjne. Baza ta została również użyta do przechowywania sesji użytkowników oraz tymczasowych statystyk aplikacji. 

Przykładowa konfiguracja użytej bazy danych:

\begin{lstlisting}[language=properties]
spring.datasource.url=jdbc:h2:mem:testdb
spring.datasource.driverClassName=org.h2.Driver
spring.datasource.username=sa
spring.datasource.password=
spring.jpa.database-platform=org.hibernate.dialect.H2Dialect
\end{lstlisting}

\section{Testy wydajnościowe}

Do oceny wydajności aplikacji przeprowadzono podstawowe testy wydajnościowe, które polegały na symulacji dużego obciążenia generowanego przez jednoczesne zapytania użytkowników. Wyniki wykazały stabilność aplikacji pod obciążeniem oraz akceptowalne czasy odpowiedzi.

Testy przeprowadzono przy użyciu narzędzi takich jak Apache JMeter. Przykładowe parametry testu:

\begin{itemize}
    \item Liczba jednoczesnych użytkowników: 100.
    \item Czas trwania testu: 10 minut.
    \item Średni czas odpowiedzi aplikacji: poniżej 300ms.
\end{itemize}

Wnioski z testów pokazały, że aplikacja działa wydajnie i jest zdolna obsłużyć przewidywaną liczbę użytkowników.

\section{Podsumowanie wyników testów}

Przeprowadzone testy jednostkowe i integracyjne potwierdziły poprawność działania aplikacji oraz jej zgodność z założonymi wymaganiami funkcjonalnymi. Testy wydajnościowe dodatkowo potwierdziły stabilność aplikacji przy dużym obciążeniu, co umożliwia jej efektywne wdrożenie w środowisku edukacyjnym.
